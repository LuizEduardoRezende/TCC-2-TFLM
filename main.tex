%% abtex2-modelo-projeto-pesquisa.tex, v-1 marcoswagner PFC 2 2023
%% Copyright 2020-2023 by abnTeX2 group at BCC-UFJ http://www.abntex.net.br/ 
%%
%% This work consists of the files abntex2-modelo-projeto-pesquisa.tex
%% and abntex2-modelo-references.bib
%%

% ---------------------------------------------------------------------
% ---------------------------------------------------------------------
% abnTeX2: Modelo Adaptado de Monografia em conformidade com 
% ABNT NBR 15287:2011 Informação e documentação - Monografia 
% --------------------------------------------------------------------- 
% ---------------------------------------------------------------------
\documentclass[
	12pt,				
	oneside,
	a4paper,		
    sumario=tradicional,
	english,			
	french,				
	spanish,			
	brazil,				
	documento
	]{abntex2}

% ---
% Configurações
% ---
\usepackage{abntex2/abntex2-bcc-ufj}
\renewcommand*\arraystretch{1.2} 
\usepackage{pdfpages} %para incluir pdf como páginas


% ---
% Pacotes básicos 
% ---
\usepackage{lmodern}					
\usepackage[T1]{fontenc}		
\usepackage[utf8]{inputenc}	
\usepackage{lastpage}		
\usepackage{indentfirst}	
\usepackage{color}			
\usepackage{graphicx}		
\usepackage{microtype} 		
% ---

% ---
% Pacotes adicionais,
\usepackage{lipsum}	
\usepackage{caption}
\usepackage{subcaption}
\usepackage{enumerate} 
\usepackage{listings}     
\usepackage{float}
\usepackage{roblang}

% --- 
% CONFIGURAÇÕES DE PACOTES
% --- 
\include{comandos}

% ---
% Informações de dados para CAPA e FOLHA DE ROSTO
% ---
% ----------------------------------------------------------
% DADOS DO TRABALHO - CAPA e FOLHA DE ROSTO
% Preencha todos os dados aqui...
% ----------------------------------------------------------
\titulo{Inferência de Modelos de Inteligência Artificial na
Robotics Language usando TensorFlow Lite Micro para
TinyML}
\autor{Luiz Eduardo Rezende Moraes}
\local{Jataí-Goiás}
\data{Dezembro de 2025}
\orientador{Prof. Dr. Thiago Borges de Oliveira}
\coorientador{}
\instituicao{%
  UNIVERSIDADE FEDERAL DE JATAÍ (UFJ)
  \par
  INSTITUTO DE CIÊNCIAS EXATAS E TECNOLÓGICAS (ICET)
  \par
  CURSO DE CIÊNCIA DA COMPUTAÇÃO}
\tipotrabalho{Monografia (Graduação)}
% O preambulo deve conter o tipo do trabalho, o objetivo, 
% o nome da instituição e a área de concentração 
\preambulo{Monografia apresentada ao curso de   Ciência da Computação do Instituto de Ciências Exatas e Tecnológicas da Universidade Federal de Jataí (UFJ), como requisito para obtenção do título de Bacharel em Ciência da Computação.}



% ---

%% \includeonly{capitulos/1RefTeorico}

%%%%%%%%%=============== DOCUMENTO ===========%%%%%%%%%%%
% ----
% Início do documento
% ----
\begin{document}
\selectlanguage{brazil}
% Retira espaço extra obsoleto entre as frases.
\frenchspacing 

% ----------------------------------------------------------
% ELEMENTOS PRÉ-TEXTUAIS
% ----------------------------------------------------------
% \pretextual

% ---
% Capa
% ---
\imprimircapa
% ---

% ---
% Folha de rosto
% ---
\imprimirfolhaderosto
% ---


% ---
% Ficha Catalográfica
\begin{fichacatalografica}
    \includepdf{ficha}
\end{fichacatalografica}


% ---
% Inserir folha de aprovação
% ---
\includepdf{aprova}


% ---
% Dedicatória/Agradecimentos/Epígrafe
% ---
\begin{dedicatoria}
   \vspace*{\fill}
   \centering
   \noindent
   \textit{
   Dedico este trabalho à minha mãe, ao meu pai e ao meu irmão, pelo apoio e incentivo durante minha jornada acadêmica e de vida. Também o dedico a todos que acreditaram em mim e me ajudaram, direta ou indiretamente, a chegar até aqui.
   } \vspace*{\fill}
\end{dedicatoria}
% ---

% ---
% Agradecimentos
% ---
\begin{agradecimentos}

 \vspace*{\fill}
   \centering
   \noindent
   \textit{
   Agradeço primeiramente à minha mãe pelo amor, apoio incondicional e sacrifícios durante toda a minha jornada acadêmica. Ao restante da minha família, pelo suporte emocional e incentivo constante. Ao meu professor e orientador Thiago Borges, pela paciência, confiança e ensinamentos que foram muito além do âmbito acadêmico. Aos demais professores, que possibilitaram meu crescimento intelectual necessário para a conclusão deste trabalho. Por fim, agradeço aos meus amigos dentro da faculdade, pelas experiências vividas que tornaram essa jornada mais fácil e divertida, e também aos meus amigos fora da faculdade, que sempre estiveram presentes para me apoiar e distrair quando necessário.
   } \vspace*{\fill}

\end{agradecimentos}
% ---

% ---
% Epígrafe
% ---
\begin{epigrafe}
    \vspace*{\fill}
    \begin{flushright}
        \textit{``Nós sempre nos definimos pela capacidade de superar o impossível.''\\
        (Cooper em INTERSTELLAR, 2014)}
    \end{flushright}
\end{epigrafe}
% ---

% ---
% RESUMOS
% ---
% ---
% RESUMOS
% ---

% resumo em português
\setlength{\absparsep}{18pt} % ajusta o espaçamento dos parágrafos do resumo
\begin{resumo}

O Tiny Machine Learning (TinyML) viabiliza a execução de modelos de aprendizado de máquina em dispositivos de baixo consumo energético, contudo, sua implementação em microcontroladores apresenta desafios significativos devido à complexidade das ferramentas de software e à necessidade de conhecimentos avançados em C++ e compilação cruzada. Este trabalho tem como objetivo democratizar o acesso à inteligência artificial embarcada integrando o framework TensorFlow Lite Micro (TFLM) à Robotics Language (RL), uma linguagem de programação de propósito específico desenvolvida na Universidade Federal de Jataí com foco na abstração de hardware e simplicidade. A metodologia adotada consistiu na implementação de um wrapper em C para garantir a interoperabilidade entre a RL e a API do TFLM, no desenvolvimento de uma biblioteca padrão (ai.tflm) e na adaptação experimental do frontend do compilador Robcmp para suporte nativo a novas sintaxes. A validação funcional foi realizada por meio de testes unitários com diversos modelos, incluindo preditores numéricos e classificadores de texto e áudio. Adicionalmente, conduziu-se uma análise comparativa de eficiência entre a solução proposta e implementações nativas em C++. Os resultados demonstraram que a integração é funcional e reduz drasticamente a complexidade do código para o desenvolvedor final, eliminando a necessidade de gerenciamento manual de tensores e operadores. Embora tenha sido observado um leve aumento no tamanho do binário e no tempo de execução devido à camada de abstração, o impacto foi considerado negligenciável frente aos ganhos em manutenibilidade e facilidade de uso. Conclui-se que a extensão da RL oferece uma alternativa viável, eficiente e mais acessível para o desenvolvimento de aplicações de TinyML em sistemas embarcados.

 \textbf{Palavras-chave}: \textit{TinyML; Sistemas Embarcados; Robotics Language; TensorFlow Lite Micro; Inteligência Artificial.} 

\end{resumo}

% resumo em inglês
\begin{resumo}[Abstract]
\begin{otherlanguage*}{english}
Tiny Machine Learning (TinyML) enables the execution of machine learning models on low-power devices; however, its implementation on microcontrollers presents significant challenges due to the complexity of software tools and the need for advanced knowledge in C++ and cross-compilation. This work aims to democratize access to embedded artificial intelligence by integrating the TensorFlow Lite Micro (TFLM) framework into Robotics Language (RL), a domain-specific programming language developed at the Federal University of Jataí focused on hardware abstraction and simplicity. The methodology involved implementing a C wrapper to ensure interoperability between RL and the TFLM API, developing a standard library (ai.tflm), and experimentally adapting the Robcmp compiler frontend for native syntax support. Functional validation was performed through unit tests with various models, including numerical predictors and text and audio classifiers. Additionally, a comparative efficiency analysis was conducted between the proposed solution and native C++ implementations. The results showed that the integration is functional and drastically reduces code complexity for the final developer, eliminating the need for manual management of tensors and operators. Although a slight increase in binary size and execution time was observed due to the abstraction layer, the impact was considered negligible compared to the gains in maintainability and usability. It is concluded that the RL extension offers a viable, efficient, and more accessible alternative for developing TinyML applications in embedded systems. 
   
\textbf{Key-words}: \textit{TinyML; Embedded Systems; Robotics Language; TensorFlow Lite Micro; Artificial Intelligence}.

\end{otherlanguage*}
\end{resumo}



% ---

% ---
% inserir lista de ilustrações
% ---
\pdfbookmark[0]{\listfigurename}{lof}
\listoffigures*
\cleardoublepage
% ---

% ---
% inserir lista de tabelas
% ---
\pdfbookmark[0]{\listtablename}{lot}
\listoftables*
\cleardoublepage
% ---

% ---
% inserir lista de abreviaturas e siglas
% ---
\begin{siglas}

\item[AF] Autômato Finito
\item[ARM] Advanced RISC Machine
\item[AST] Árvore Sintática Abstrata
\item[AVR] Alf and Vegard’s RISC processor
\item[CFG] Context-Free Grammar
\item[CPU] Central Processing Unit
\item[IA] Inteligência Artificial
\item[IoT] Internet of Things
\item[IR] Intermediate Representation
\item[MCU] Microcontroller Unit
\item[ML] Machine Learning
\item[RAM] Random access memory
\item[RE] Regular Expression
\item[RL] Robotics Language
\item[TFLM] TensorFlow Lite Micro

\end{siglas}
% ---


% ---
% inserir o sumario
% ---
\pdfbookmark[0]{\contentsname}{toc}
\tableofcontents*
\cleardoublepage
% ---

% ===================================================
% ELEMENTOS TEXTUAIS - Capítulos ----
% ----------------------------------------------------------
\textual

\chapter[Introdução]{Introdução}\label{cap:introducao}

\section{Motivação (objeto de estudo e problema)}
O Machine Learning (ML) está se tornando cada vez mais comum em diversas áreas da sociedade, como em veículos autônomos, reconhecimento facial e monitoramento de saúde. \textbf{Comentário Emília: Sem referências.}Esse recurso inovador pode estar presente não somente em supercomputadores, celulares e robôs, mas também em sistemas embarcados. Aplicações nessas plataformas já são uma realidade, como apresentado por \citeonline{RoadMarkings}, que implementou um modelo de Inteligência Artificial (IA) para melhorar a identificação de marcações viárias em carros autônomos, e por \citeonline{8274968}, que utilizou circuitos dedicados e plataformas embarcadas para detectar o estresse humano. No entanto, a adoção de ML em microcontroladores de baixo custo ainda é um campo pouco explorado, devido a limitações de \textit{hardware} e falta de ferramentas intuitivas para o desenvolvimento. \textbf{Comentário Emília: De acordo com quem ?}

Ao contrário do que se imagina, microcontroladores (MCU, do inglês \textit{Microcontroller Unit}) podem ser encontrados em diversos aparelhos cotidianos como eletrodomésticos, automóveis, dispositivos de saúde e lâmpadas inteligentes. Qualquer dispositivo que meça, armazene, controle, calcule ou exiba informações é um candidato a ter um microcontrolador embutido \cite{axelson1997microcontroller}. Eles podem ser definidos como um computador em um único circuito integrado que inclui uma \textit{Central Processing Unit} (CPU), \textit{Random Access Memory} (RAM), alguma forma de \textit{Read-Only Memory} (ROM) e portas de \textit{Input/Output} (I/O) \cite{hussain2016programming}. Frequentemente, o termo pode ser utilizado como sinônimo para sistema embarcado, porém existe uma distinção clara entre os dois conceitos. Enquanto o MCU refere-se apenas ao \textit{hardware}, o sistema embarcado compreende o sistema completo (\textit{hardware} + \textit{firmware}), já pronto para uso.

O uso de sistemas embarcados também está fortemente ligado à ideia de \textit{Internet of Things} (IoT), uma rede de dispositivos eletrônicos de baixo custo onde a comunicação e a coleta de dados ocorrem automaticamente, por meio de protocolos de comunicação \cite{microcontrolerforIoT}. Atualmente, esse tipo de sistema já é amplamente utilizado para captura e processamento de dados em projetos de IoT como cidades inteligentes, automação residencial e agricultura de precisão.\textbf{Comentário Emília: Sem referências.} Quando o projeto envolve IA, após a captura dos dados, eles devem ser enviados para essa IA para uma tomada de decisão. Isso pode ocorrer de duas formas: utilizando \textit{Cloud AI} ou \textit{Edge AI}. 

\textit{Edge AI} (Inteligência Artificial na Borda) refere-se à prática de realizar computações de IA próximo aos usuários na borda da rede, em vez de em localizações centralizadas como os \textit{data centers} de provedores de serviços em nuvem (\textit{Cloud AI}) \cite{SINGH202371}. Segundo \citeonline{SINGH202371}, \textit{Edge AI} se destaca em situações que precisam de baixa latência, privacidade e eficiência energética. Já a \textit{Cloud AI} é mais indicada para tarefas que exigem um grande poder de processamento e muito espaço de armazenamento. O uso de ML em MCUs se enquadra no conceito de IA na borda, mas em um campo ainda mais específico e emergente: o \textit{Tiny Machine Learning} (TinyML).

\textbf{Comentário Emília: Citação direta deve ter recuo. (somente citações com mais de 3 linhas devem ter recuo).}

De acordo com \citeonline{warden2019tinyml}, ``Se você consegue executar um modelo de rede neural com um custo energético abaixo de 1 mW, isso torna possíveis muitas aplicações completamente novas''. Essa é a principal premissa e também definição do termo TinyML, desse modo qualquer aplicação que consiga executar um modelo a um custo de energia baixíssimo pode ser considerada como TinyML. Esse número pode parecer arbitrário (1mW), mas, em termos concretos, significa que um dispositivo alimentado por uma bateria de botão pode funcionar por um ano. Isso resulta em um produto que é pequeno o suficiente para ser instalado em qualquer ambiente e capaz de operar por um tempo útil sem intervenção humana \cite{warden2019tinyml}.

Para a execução de TinyML, além dos microcontroladores convencionais, podem ser utilizados outros circuitos integrados (CI), como ASICs (\textit{Application-Specific Integrated Circuits}), FPGAs (\textit{Field-Programmable Gate Arrays}) e DSPs (\textit{Digital Signal Processors}). Apesar da existência de diferentes opções, os MCUs mostram-se mais vantajosos por geralmente serem menores, apresentarem baixíssimo consumo de energia e possuírem custo extremamente baixo. Embora o \textit{hardware} especializado proporcione o melhor desempenho para TinyML, microcontroladores de propósito geral oferecem maior flexibilidade, como discutido por \citeonline{abadade2023comprehensive}.

Devido à falta de poder computacional, executar modelos de ML em sistemas embarcados não é uma tarefa fácil; no entanto, segundo \citeonline{soro2021tinyml}, existem três formas principais de realizar esse feito:

\begin{itemize}
    \item Codificação Manual, implementação manual do modelo de ML diretamente em código de baixo nível, como C ou C++. É uma maneira trabalhosa, demorada e propensa a erros, devido ao grau de complexidade.
    \item Geração automática de código, uso de ferramentas para converter um modelo pré-treinado em código otimizado para o dispositivo embarcado. A ferramenta converte, por exemplo, um arquivo .tflite em código C ou C++.
    \item Interpretador de ML, é utilizado um interpretador para executar o modelo diretamente no dispositivo.
\end{itemize}

Várias ferramentas para TinyML já existem no mercado, algumas são produzidas por grandes empresas como a Google, outras são desenvolvidas por grupos menores de desenvolvedores. 
Cada uma delas possui em sua estrutura uma técnica de geração automática de código ou um interpretador de ML. O objetivo de um \textit{framework} TinyML é fornecer uma solução abrangente para a construção e implantação de modelos de ML em dispositivos de baixo consumo de energia, facilitando o desenvolvimento de aplicações de \textit{edge computing} por desenvolvedores \cite{abadade2023comprehensive}.

\textbf{Comentário Ariadne: Na redação final do TCC2, sugiro que inclua discussões mais amplas sobre o impacto social, educacional e de mercado do uso do TinyML em MCUs de baixo custo com linguagem própria.}

Desenvolvido pela Google, o TensorFlow Lite\footnote{Documentação do TensorFlow Lite: \url{https://www.tensorflow.org/lite/guide?hl=pt-br}} foi lançado em 2017 e se destacou como um dos pioneiros em \textit{Edge AI}, tendo grande importância na democratização do ML em dispositivos limitados. Atualmente, seu nome foi alterado para LiteRT\footnote{Documentação do Lite RT: \url{https://ai.google.dev/edge/litert?hl=pt-br}}, um produto oficial da Google AI Edge, uma mudança estratégica da empresa para unificar ferramentas de ML no dispositivo. Agora, o LiteRT consegue cobrir múltiplos \textit{frameworks} de treinamento, como JAX\footnote{Documentação do JAX: \url{https://docs.jax.dev/en/latest/}}, Keras\footnote{Site do Keras: \url{https://keras.io}}, PyTorch\footnote{Site do PyTorch: \url{https://pytorch.org}} e TensorFlow. Apesar das atualizações no nome e na amplitude do produto, a ferramenta oficial da Google para desenvolvimento de TinyML em microcontroladores teve seu nome mantido como TensorFlow Lite Micro (TFLM).

Conforme o trabalho de \citeonline{MLSYS2021_6c44dc73}, o \textit{framework} trabalha com alocação estática de memória, possui um interpretador de ML com menos de 2 kB e já foi validado em diversas arquiteturas, como Arm Cortex-M, ESP32 e DSPs. Sua abordagem baseada em interpretador (que lê o modelo .tflite e executa as operações nele contidas) viabiliza a atualização de modelos ``em campo'', ou seja, em dispositivos que já estão em posse do usuário final ou instalados em seu local de operação. Isso ocorre porque o \textit{firmware} pode ser programado para carregar o modelo de uma área de memória separada, como uma partição Flash, permitindo a substituição do arquivo do modelo sem a necessidade de recompilar o \textit{firmware}.

No contexto dos sistemas embarcados, o projeto Robotics Language (PI05974-2024), realizado na Universidade Federal de Jataí (UFJ), desenvolve uma linguagem de programação especializada no campo da robótica e dos microcontroladores. Essa iniciativa é uma evolução de um projeto anterior que lidava com kits robóticos educacionais (PI02361-2018) e busca, de maneira geral, tornar a programação em microcontroladores mais simples e acessível.

Para realizar tal objetivo, a linguagem busca isolar as especificidades de microcontroladores, fornecendo uma camada de abstração de \textit{hardware} dentro do próprio compilador, em vez de depender de \textit{frameworks} ou bibliotecas externas. Essa abordagem permite que os desenvolvedores escrevam o código uma única vez, eliminando a necessidade de adaptar o código para cada \textit{hardware} através de macros condicionais. Ao aproveitar a análise semântica do compilador, é possível prevenir erros comuns encontrados no desenvolvimento de \textit{firmware} em linguagens não específicas de domínio, como C/C++ \cite{borges_robcmp}.

\section{Objetivo do Trabalho}
Este trabalho teve como objetivo integrar o TFLM à linguagem, estendendo sua sintaxe para viabilizar a execução de modelos de ML nos microcontroladores já compatíveis com o \textit{framework} e suportados pela linguagem. Os objetivos específicos foram:
\begin{itemize}
    \item Adicionar novas palavras-chave, regras sintáticas e semânticas na linguagem RL para a inferência de modelos de TinyML;
    \item Implementar um \textit{wrapper} parcial da API TFLM em C, para possibilitar a ligação do TFLM com o código compilado em RL;
    \item Validar a integração por meio de testes funcionais com modelos de IA existentes e já treinados; e
    \item Realizar uma análise comparativa de eficiência entre a solução em RL e uma implementação nativa em C++.
\end{itemize}

\section{Contribuição do Trabalho}
A principal contribuição deste trabalho é a extensão da RL para viabilizar o desenvolvimento de aplicações de ML em microcontroladores de forma nativa. Ao integrar o \textit{framework} TFLM, este projeto dota a linguagem de uma sintaxe de alto nível que abstrai a complexidade da inferência. Dessa forma, os desenvolvedores podem aliar o poder do TinyML aos benefícios já oferecidos pela RL, como o desenvolvimento de firmware com baixo acoplamento, alta coesão e elevada manutenibilidade.

\textbf{Comentários Emília: Qual o seu diferêncial para com a literatura?, Qual sua contribuição?, Qual a justificativa da pesquisa?}

\textbf{Sugestão de hipotese: É possível integrar o TensorFlow Lite Micro (TFLM) à Robotics Language (RL), estendendo sua sintaxe e
utilizando um wrapper em C, de forma que seja viável executar modelos de Machine Learning (ML) em
microcontroladores de baixo custo (TinyML) de maneira mais simples, privada e com menor latência, e
que essa solução apresente eficiência comparável à implementações nativas em C++}

% Esqueleto Inicial do modelo de TCC 2
% \section{Motivação (objeto de estudo e problema)}
% \section{Objetivo do Trabalho}
% \section{Referencial Teórico Resumido}
% \section{Contribuição do Trabalho}
% \section{Organização da Monografia}





\chapter{Referencial Teórico}\label{cap:referencial}

Este capítulo tem como objetivo principal apresentar os conceitos teóricos que são amplamente utilizados ao longo deste trabalho. Ele se mostra fundamental para que seja constuída uma base sólida de conhecimentos e conceitos essenciais, os quais são necessários para compreender desde a comparação entre os trabalhos relacionados até a implementação e também a validação da solução proposta. 

Inicialmente é apresentada uma visão sobre o funcionamento de um compilador, seguida por uma introdução aos sistemas embarcados e microcontroladores. Em seguida, é abordado o conceito de IA, com foco em TinyML e seus \textit{frameworks}. Por fim, são discutidos os conceitos de interoperabilidade entre linguagens de programação e a RL com seu compilador Robcmp.

\section{Compilador}

Compiladores são programas com uma função aparentemente simples: traduzir código de uma linguagem para outra. No entanto, por trás dessa definição sucinta, está um dos sistemas mais complexos da computação, formado por muitos componentes internos, algoritmos e interações complexas entre eles \cite{cooper2013engineering}.

Inicialmente, o compilador pode ser visto como uma caixa-preta que recebe como entrada um código fonte (ou programa fonte) e o transforma em um código objeto (ou programa objeto). A \hyperref[fig:compilador]{Figura~\ref*{fig:compilador}} ilustra esse esquema simplificado, que ainda desconsidera as etapas internas do processo de compilação. O código fonte pode ser escrito em diversas linguagens, como C, C++, Fortran, Java ou até mesmo a RL. Já a linguagem do código objeto corresponde ao conjunto de instruções de uma CPU ou MCU específico.

\begin{figure}[H]
\centering
\includegraphics[width=0.5\linewidth]{Imagens/Compilador.png} % Ajuste o valor entre 0-1
\caption{Esquema simplificado de um compilador. Adaptado de \citeonline{cooper2013engineering}.}
\label{fig:compilador}
\end{figure}

É essencial conhecer as etapas envolvidas no processo de um compilador. As duas principais divisões existentes são: \textit{frontend} e \textit{backend}. Essa divisão de tarefas pode ser visualizada na \hyperref[fig:compilador-front-back]{Figura~\ref*{fig:compilador-front-back}}, que sintetiza a estrutura do processo de compilação. O \textit{frontend} do compilador é responsável por analisar o programa fonte, incluindo o processamento léxico, sintático e semântico, resultando em uma representação intermediária, como uma Árvore Sintática Abstrata (AST, do inglês \textit{Abstract Syntax Tree}) ou um código intermediário. O \textit{backend} recebe essa representação intermediária e gera código alvo otimizado, lidando com detalhes específicos da máquina, como alocação de registradores e seleção de instruções \cite{fischer2010crafting}.

\begin{figure}[H]
\centering
\includegraphics[width=0.7\linewidth]{Imagens/Compilador-front-back.png} % Ajuste o valor entre 0-1
\caption{Compilador dividido em duas etapas. Adaptado de \citeonline{cooper2013engineering}.}
\label{fig:compilador-front-back}
\end{figure}

\section{\textit{Frontend}}
Para que o compilador consiga traduzir o código do programa fonte, ele precisa entender tanto a forma (sintaxe) quanto o significado (semântica). O \textit{frontend} é a primeira etapa da compilação e é responsável por determinar se o código está bem construído em termos de sintaxe e semântica. Se ele identificar um código válido, uma representação intermediária é criada; caso contrário, reporta o erro ao usuário para que o problema seja identificado \cite{cooper2013engineering}.

A \hyperref[fig:front-end]{Figura~\ref*{fig:front-end}} ilustra as principais etapas do \textit{frontend} consideradas neste estudo. Inicialmente, o código-fonte é submetido ao analisador léxico, que converte a sequência de caracteres em \textit{tokens} (unidades significativas). Em seguida, o analisador sintático verifica a estrutura desses \textit{tokens} e gera uma AST. Essa AST é então utilizada pelo analisador semântico para produzir uma Árvore Sintática Decorada, que finalmente é processada para gerar a Representação Intermediária (IR, do inglês \textit{Intermediate Representation}). Todas essas etapas utilizam a tabela de símbolos, seja para armazenar novas entradas ou para verificar a existência de símbolos já declarados.

\begin{figure}[H]
\centering
\includegraphics[width=1\linewidth]{Imagens/Front-end.png} % Ajuste o valor entre 0-1
\caption{Etapas do \textit{frontend} de um compilador. Adaptado de \citeonline{fischer2010crafting}.}
\label{fig:front-end}
\end{figure}

\subsection{Analisador Léxico}
A primeira etapa do \textit{frontend} consiste em um analisador Léxico, ou \textit{Scanner} em inglês. Ele inicia a análise do programa fonte lendo o texto de entrada (caractere por caractere) e agrupando caracteres individuais em \textit{tokens}, como identificadores, números inteiros, palavras reservadas e delimitadores \cite[p. 16]{fischer2010crafting}. No analisador léxico, Autômatos Finitos (AFs) atuam como núcleo do processo de reconhecimento de \textit{tokens}. Eles simulam um fluxo de transições entre estados, guiados pelos caracteres de entrada, até determinar se uma sequência é válida (aceita) ou inválida (rejeitada) \cite[p. 23]{cooper2013engineering}.

O conjunto de palavras aceitas por um autômato finito, F, forma uma linguagem, indicada por L(F). Essa linguagem é definida de maneira precisa pelo seu diagrama de transição, que descreve todas as possíveis sequências de estados e símbolos aceitos pelo AF. Para qualquer AF, também podemos descrever sua linguagem usando uma notação chamada expressão regular (RE, do inglês \textit{Regular Expression}). A linguagem descrita por uma RE é chamada de linguagem regular \cite[p. 26]{cooper2013engineering}. Tanto um AF quanto uma RE podem ser usados em um programa gerador de \textit{scanner}, um programa que produz efetivamente um analisador léxico funcional. Geradores de analisadores léxicos são ferramentas valiosas para a construção de compiladores \cite[p. 16]{fischer2010crafting}.

\subsection{Analisador Sintático}
A principal tarefa do analisador sintático (ou \textit{parser}, em inglês) é verificar se o programa de entrada constitui uma sentença sintaticamente válida na linguagem-fonte. Para essa finalidade, essa etapa utiliza gramáticas livres de contexto, uma vez que REs não são mais suficientes para descrever a sintaxe complexa presente na maioria das linguagens de programação \cite[p. 69, 70]{cooper2013engineering}.

Apesar de uma RE, como a apresentada na \hyperref[fig:RE]{Figura~\ref*{fig:RE}}, poder reconhecer a expressão a + b × c corretamente, ela não pode especificar a ordem de precedência entre os operadores. Conforme as regras algébricas convencionais, as operações de multiplicação e divisão possuem precedência sobre adição e subtração. Essa ordem de avaliação pode também ser alterada utilizando parênteses como mecanismo de agrupamento. Porém, a simples inclusão de parênteses em REs não resolve adequadamente o problema de precedência, pois é impossível definir uma RE que reconheça todas as possíveis combinações de parênteses balanceados \cite[p. 71]{cooper2013engineering}. Em outras palavras, a linguagem reconhecida por uma RE não é suficiente para o reconhecimento sintático do compilador.

\begin{figure}[H]
\centering
\includegraphics[width=0.8\linewidth]{Imagens/RE.png} % Ajuste o valor entre 0-1
\caption{RE utilizada para identificar identificadores compostos e operações aritméticas simples entre identificadores, formadas por letras minúsculas, dígitos e operadores (+, -, x, ÷). Adaptado de \citeonline{cooper2013engineering}.}
\label{fig:RE}
\end{figure}

Em vez de utilizar REs para verificar sentenças, o analisador sintático emprega uma gramática livre de contexto (CFG, do inglês \textit{Context-Free Grammar}), um conjunto de regras que descreve como as sentenças podem ser formadas. A coleção de todas as sentenças deriváveis a partir de G é denominada linguagem definida por G, representada por L(G) \cite[p. 71]{cooper2013engineering}. Esse mecanismo demonstra maior poder expressivo e complexidade na especificação de construções sintáticas, uma vez que, diferentemente das REs, possui algumas capacidades adicionais que serão destacadas a seguir.

Uma CFG pode introduzir precedência criando níveis separados na gramática (Expr, Term, Factor). Operadores de maior precedência aparecem em níveis mais internos e, por isso, são reduzidos antes dos de menor precedência. 

No exemplo do \hyperref[lst:cfg-precedencia]{Código~\ref{lst:cfg-precedencia}}, a adição e subtração são definidas em expr, enquanto multiplicação e divisão ficam em term e são resolvidas antes de + e - que ficam em expr. A recursão à esquerda (expr → expr op term e term → term op factor) impõe associatividade à esquerda: a - b - c é agrupado como (a - b) - c. Se a gramática tivesse recursão à direita (expr → term ('+' expr | '-' expr)), o agrupamento seria a - (b - c). Parênteses em factor permitem sobrescrever essa ordem.

\[
a - b - c \equiv (a - b) - c \qquad a / b / c \equiv (a / b) / c
\]

\begin{lstlisting}[caption={Gramática (formato Bison) que define precedência e associatividade para operadores aritméticos},label={lst:cfg-precedencia}]
expr : expr '+' term
     | expr '-' term
     | term
     ;

term : term '*' factor
     | term '/' factor
     | factor
    ;

factor : '(' expr ')'
       | TOK_IDENT
       | TOK_INTEIRO
       ;
\end{lstlisting}


Além disso, uma CFG pode validar construções com parênteses balanceados; A regra Expr -> ( Expr ) na \hyperref[fig:CFG-regra]{Figura~\ref{fig:CFG-regra}} demonstra essa capacidade, permitindo que expressões sejam aninhadas recursivamente dentro de parênteses. O analisador sintático utiliza essa regra para garantir que, para cada parêntese de abertura, exista um correspondente de fechamento.

Por fim, a última capacidade de uma CFG que será abordada é sua representação de hierarquias sintáticas através de árvores de derivação. Na \hyperref[fig:CFG-arvore]{Figura~\ref{fig:CFG-arvore}}, a árvore ilustra a estrutura (a + b) × c. Isso ocorre porque a regra Expr -> Expr Op nome força uma avaliação da associatividade à esquerda. A árvore mostra a operação de adição a + b sendo formada em um nível mais baixo e, em seguida, o resultado dessa subárvore é usado como operando para a multiplicação por c, que está no topo da árvore.

\begin{figure}[H]
\centering
\includegraphics[width=0.8\linewidth]{Imagens/CFG-regras.png} % Ajuste o valor entre 0-1
\caption{Regras de derivação de uma CFG. Adaptado de \citeonline{cooper2013engineering}.}
\label{fig:CFG-regra}
\end{figure}

\begin{figure}[H]
\centering
\includegraphics[width=0.8\linewidth]{Imagens/CFG-arvore.png} % Ajuste o valor entre 0-1
\caption{Árvore sintática de uma CFG. Adaptado de \citeonline{cooper2013engineering}.}
\label{fig:CFG-arvore}
\end{figure}

Como resultado da análise sintática, constrói-se uma AST utilizando as informações obtidas do programa fonte e do analisador léxico, assegurando a validação da estrutura sintática do código. Simultaneamente, a tabela de símbolos é atualizada com metadados essenciais sobre cada identificador, incluindo seu tipo de dado, o tamanho de sua representação em memória durante a execução e, no caso de \textit{arrays}, o número de dimensões e os intervalos indexados de cada dimensão \cite[p. 176]{cooper2013engineering}.

\subsection{Análise Semântica}

A Análise Semântica, como o próprio nome indica, tem como objetivo examinar o significado (semântica) do programa fonte. Para isso, ela utiliza a AST construída pelo Analisador Sintático, juntamente com as informações contidas na tabela de símbolos. Uma das principais funções da análise semântica é a verificação de tipos, na qual o compilador verifica a compatibilidade entre os operandos de cada operador \cite{aho2008compiladores}.

Entre as regras semânticas mais comuns destacam-se:
\begin{itemize}
    \item A exigência de que índices de \textit{arrays} sejam do tipo inteiro;
    \item A restrição de que operandos em operações aritméticas possuam tipos compatíveis;
    \item A proibição de redeclaração de variáveis no mesmo escopo;
    \item A verificação de que expressões condicionais retornem valores booleanos.
\end{itemize}

No final dessa fase, temos como resultado a Árvore Sintática Decorada, uma versão aprimorada da AST original. A grande diferença é que agora ela traz consigo uma informação importante: os tipos de dados de todos os identificadores usados no programa. Essas informações adicionais são vitais para as fases seguintes, especialmente para a geração de código, onde é necessário saber exatamente qual é o tipo de cada variável e expressão.

\subsection{Geração de Código Intermediário}

Durante todo o processo de compilação, podem ser produzidas uma ou mais IRs, as quais podem assumir diversas formas. As árvores sintáticas constituem um tipo de IR, assim como outros formatos de código em baixo nível ou semelhantes à linguagem de máquina. Esses códigos intermediários devem possuir duas propriedades essenciais: precisam ser facilmente gerados e simples de traduzir para a linguagem da máquina alvo \cite{aho2008compiladores}.

O Robcmp utiliza o \textit{backend} da infraestrutura LLVM; por essa razão, sua IR segue o padrão LLVM-IR. O LLVM consiste em uma coleção modular e reutilizável de tecnologias de compilador, que proporciona um otimizador moderno e independente de alvo, além de suporte para a geração de código para diversas CPUs populares.

A representação de código do LLVM foi projetada para ser utilizada de três formas distintas: como uma IR de compilador em memória, como uma representação em bitcode armazenada em disco (adequada para carregamento rápido por um compilador \textit{Just-In-Time}) e como uma linguagem de montagem de alto nível, legível por humanos \cite{LLVMLangRef}.

A \hyperref[fig:LLVM-IR]{Figura~\ref{fig:LLVM-IR}} ilustra uma IR que realiza três operações sequenciais: (1) define uma variável contendo a \textit{string} ``hello world'', (2) declara a função externa ``puts'', responsável por imprimir \textit{strings} na saída padrão, e (3) implementa a função ``main'', que invoca ``puts'' para exibir a mensagem armazenada.

\begin{figure}[H]
\centering
\includegraphics[width=0.8\linewidth]{Imagens/LLVM-IR.png} % Ajuste o valor entre 0-1
\caption{Exemplo de LLVM-IR de um código ``hello world''. Adaptado de \citeonline{LLVMLangRef}.}
\label{fig:LLVM-IR}
\end{figure}

\section{Backend}

O papel do \textit{backend} é receber o código IR e produzir um código para a máquina-alvo. Ele seleciona as operações específicas da máquina-alvo para implementar cada operação da IR, determina a ordem mais eficiente de execução dessas operações e decide quais valores serão armazenados em registradores e quais residirão na memória \cite[p. 13]{cooper2013engineering}. Essas decisões podem ser ajustadas pelo desenvolvedor por meio de flags de otimização, como -Oz (otimização para tamanho extremo) e -Os (otimização para tamanho).

Uma das vantagens do \textit{backend} da LLVM é seu suporte a diversas plataformas, incluindo arquiteturas de CPUs (como x86 e ARM) e famílias de MCUs (como AVR e STM32, ambas baseadas em diferentes arquiteturas). Isso permite que a RL também tenha portabilidade para um amplo espectro de \textit{hardware}. Por exemplo, é possível compilar código para microcontroladores específicos como o ATmega328P (da família AVR, baseada na arquitetura AVR) ou o STM32F103C8T6 (da família STM32, baseada na arquitetura ARM Cortex-M). Estes microcontroladores frequentemente compõem sistemas embarcados, como será detalhado a seguir.


\section{Sistemas Embarcados}
Segundo \citeonline{marwedel2021embedded}, os sistemas embarcados são sistemas de processamento de informação embutidos nos produtos que os contêm, ou seja, são parte de um sistema maior. Atualmente, estão presentes em inúmeras aplicações, como carros, trens, aviões, equipamentos industriais e de telecomunicações.

Apesar da definição abrangente, é possível obter muitas informações a partir dessa descrição. Como um sistema embarcado faz parte de um sistema maior, ele é frequentemente dedicado a uma tarefa específica. Desse modo, processadores que controlam determinados sistemas de carros ou trens, por exemplo, sempre executam o mesmo \textit{software} \cite[p. 17]{marwedel2021embedded}. Esse \textit{software}, por ser o programa fundamental que gerencia o \textit{hardware} para essas funções específicas, é corretamente classificado como \textit{firmware}. 

Além de serem parte fundamental de dispositivos individuais, esses componentes também formam o núcleo de aplicações de IoT. De acordo com \citeonline{Giusto2010IoT}, o termo \textit{Internet of Things} é um paradigma que abrange objetos físicos que se conectam à Internet para compartilhar informações. Tais informações podem ser capturadas por sensores, processadas por um microcontrolador e repassadas a outros dispositivos, a fim de que atinjam um objetivo em comum.

Existe um grande potencial para aplicações de processamento de informações no contexto de IoT e sistemas embarcados. Sua enorme abrangência inclui áreas como eletrônica automotiva, aeronáutica, transporte ferroviário, engenharia marítima, engenharia mecânica, robótica, engenharia civil, recuperação de desastres, engenharia agrícola, setor de saúde, entre outras \cite{marwedel2021embedded}.

Para que o processamento de informações ocorra, os dados devem ser processados por uma CPU. Nos sistemas embarcados, MCUs são frequentemente empregados para essa tarefa. A CPU integrada a um MCU não só permite executar todas as operações necessárias, como também possibilita que o sistema se beneficie das vantagens inerentes a um microcontrolador.

\section{Microcontroladores}

O núcleo da maioria dos Sistemas Embarcados é o seu centro de processamento, constituído por um MCU. Segundo \citeonline{hussain2016programming}, microcontroladores são computadores em um único chip: um circuito que integra uma unidade central de processamento, memória de acesso aleatório, alguma forma de memória apenas de leitura e portas de entrada/saída.

Ao contrário dos computadores de uso geral (como computadores pessoais, \textit{smartphones} e \textit{tablets}), os MCUs são dedicados a tarefas específicas e executam uma única aplicação \cite{hussain2016programming}. O MCU é um componente eletrônico único; já os sistemas embarcados são o conjunto completo, que inclui, além do próprio MCU, outros elementos como sensores, atuadores, fontes de alimentação e interfaces de comunicação, todos trabalhando em conjunto para realizar uma função determinada.

Algumas das famílias de microcontroladores mais comuns e que também possuem microcontroladores suportados pela RL são:

\begin{itemize}

    \item AVR: É uma família de microcontroladores de 8 bits, baseada na arquitetura AVR, originalmente desenvolvida pela Atmel, empresa fundada em 1984 e especializada em semicondutores para aplicações embarcadas \cite{Atmel_History}. Atualmente, os microcontroladores AVR são produzidos pela Microchip Technology. Estes microcontroladores combinam desempenho eficiente com baixo consumo de energia, oferecendo flexibilidade para uma vasta gama de aplicações embarcadas \cite{Microchip_AVR_MCUs}. A \hyperref[fig:ATmega328PB]{Figura~\ref{fig:ATmega328PB}} apresenta um exemplo de MCU dessa família.

    \item STM32: É uma família de microcontroladores de 32 bits fabricada pela STMicroelectronics, baseada na arquitetura ARM Cortex-M. Os MCUs STM32 são otimizados para eficiência energética e operações determinísticas, sendo uma alternativa atraente às arquiteturas de MCU de propósito geral de 8 bits e 16 bits \cite{ST_Arm_MCUs_2025}. A \hyperref[fig:STM32F103C8T6]{Figura~\ref{fig:STM32F103C8T6}} mostra uma placa de desenvolvimento que contém um MCU dessa família.


\end{itemize}

\begin{figure}[H]
\centering
\includegraphics[width=0.3\linewidth]{Imagens/ATmega328PB.png} % Ajuste o valor entre 0-1
\caption{Microcontrolador de 8 bits ATmega328PB, anteriormente produzido pela Atmel, atualmente fabricado pela Microchip Technology como parte da família AVR. Possui 32KB de memória Flash, 2KB de SRAM e 27 portas de entrada/saída \citeonline{Microchip_AVR_MCUs}.}
\label{fig:ATmega328PB}
\end{figure}

\begin{figure}[H]
\centering
\includegraphics[width=0.3\linewidth]{Imagens/STM32F103C8T6.jpg} % Ajuste o valor entre 0-1
\caption{Placa de desenvolvimento com o MCU STM32F103C8T6, que possui 72 MHz de frequência máxima, 20 kB de memória SRAM, 128 kB de memória FLASH e 37 portas de entrada/saída \cite{STMicroelectronics:STM32F103x8_xB_2023}.}
\label{fig:STM32F103C8T6}
\end{figure}

O avanço dos microcontroladores e das tecnologias de desenvolvimento tem permitido a incorporação de funcionalidades inteligentes diretamente em dispositivos embarcados. Essa evolução cria uma conexão natural com o campo da Inteligência Artificial, tema que será abordado na próxima seção.

\section{Inteligência Artificial}

Na última década, o assunto IA tornou-se popular e impossível de evitar em canais midiáticos como televisão, redes sociais e até mesmo em conversas cotidianas. Pode-se dizer que ele se tornou um termo \textit{mainstream}, expressão do inglês que designa um conteúdo considerado normal que é aceito pela maioria das pessoas \cite{Cambridge2024}. Essa tecnologia ganhou destaque devido ao seu rápido crescimento e aplicação em diversas áreas da sociedade, consolidando-se como uma inovação promissora que marcou o início de uma nova era a partir da década de 2020.

Apesar de ter se tornado comum e de fácil compreensão em discussões informais, o conceito de IA é frequentemente confundido com ML, como se fossem sinônimos. Segundo \citeonline{russell2020artificial}, o ramo da IA preocupa-se em estudar e construir entidades inteligentes capazes de computar como agir eficientemente em um amplo espectro de situações. Além disso, a IA atualmente abrange diversas subáreas, que vão desde capacidades gerais (como aprendizagem, raciocínio e percepção) até tarefas específicas, tais como jogar xadrez \cite{stockfish}, provar teoremas matemáticos \cite{gptf_math_theorem}, dirigir veículos \cite{tesla_autonomous}, ou diagnosticar doenças \cite{8274968}. Por essa razão, a IA constitui um campo relevante para qualquer tarefa que exija raciocínio intelectual, abrangendo inúmeros domínios do conhecimento.

Enquanto a IA possui um espectro mais abrangente e um objetivo filosófico mais amplo, o ML é um subcampo da IA que estuda como melhorar o desempenho com base na experiência. Ou seja, utiliza dados para realizar um treinamento, permitindo que o sistema computacional tome decisões baseadas em experiências passadas obtidas desses dados. Alguns sistemas de IA empregam métodos de ML, mas outros não \cite{russell2020artificial}.

O resultado de treinamentos utilizando grandes bases de dados são modelos de ML, que podem ser executados em diferentes tipos de computadores, dependendo do \textit{hardware} requerido pelo modelo. Esses modelos podem ser executados na borda da rede (\textit{Edge AI}), isto é, em celulares, computadores pessoais e sistemas embarcados, ou na nuvem (\textit{Cloud AI}), isto é, em \textit{data centers} e servidores, para que o modelo possa ser acessado via Internet.

Segundo \citeonline{SINGH202371}, \textit{Edge AI} se destaca em situações que precisam de baixa latência, privacidade e eficiência energética. Já a \textit{Cloud AI} é mais indicada para tarefas que exigem um grande poder de processamento e muito espaço de armazenamento. O uso de ML em MCUs se enquadra no conceito de IA na Borda, mas em um campo ainda mais específico e emergente: o TinyML, que é o foco principal deste trabalho.

Um exemplo de modelo aplicável a TinyML é o Micro Speech\footnote{\url{https://github.com/tensorflow/tflite-micro/tree/main/tensorflow/lite/micro/examples/micro_speech}}, um modelo exemplo do repositório oficial do TFLM. O exemplo Micro Speech demonstra a execução de inferência para reconhecimento de palavras-chave (``sim'' e ``não''). O sistema consiste em dois modelos principais:

\begin{itemize}
    \item Pré-processador de áudio: Converte amostras de áudio bruto em dados espectrográficos; e
    \item Modelo Micro Speech: Modelo compacto (<20 kB) que classifica as palavras-chave a partir dos espectrogramas.
\end{itemize}

Este projeto irá expandir a RL para suportar a execução de modelos de ML nos MCUs suportados pela linguagem. Dessa forma, será possível realizar inferências com maior privacidade e menor latência, uma vez que não será necessário acessar modelos hospedados em servidores na Internet, mantendo os dados estritamente no âmbito local.



\subsection{Engines de Inferência}

\textit{Engines} de Inferência (ou Motores de Inferência) são mecanismos que permitem a execução de um modelo de ML já treinado. Na fase de inferência, esse modelo é implementado, e o motor de inferência utiliza a ``inteligência'' aprendida para compreender e fazer previsões ou classificações sobre dados novos e não vistos.

Geralmente, os mecanismos de inferência fazem parte de um \textit{framework} maior, que contém ferramentas tanto para o treinamento quanto para a inferência de modelos. Entre os exemplos mais famosos estão TensorFlow, PyTorch, Caffe, Keras e JAX.

Esses \textit{frameworks} podem ser usados para treinar desde modelos simples até os mais complexos, mas a escolha do \textit{hardware} para o treinamento depende da necessidade de cada modelo. Enquanto modelos simples podem ser treinados utilizando computadores comuns, modelos mais complexos, como os de \textit{Deep Learning} (Aprendizagem Profunda), são muito mais exigentes e precisam ser treinados em computadores equipados com Unidades de Processamento Gráfico (GPUs, do inglês \textit{Graphics Processing Units}) potentes ou especializadas.

\subsection{Engines para o TinyML}

Considera-se TinyML qualquer aplicação capaz de executar modelos de ML com um consumo de energia extremamente baixo \cite{warden2019tinyml}. Nesse contexto, diferentemente do contexto geral de ML, os \textit{frameworks} disponíveis para o desenvolvimento possuem, na maioria das vezes, somente os mecanismos para inferência. Afinal, o \textit{hardware} de dispositivos como um microcontrolador não é suficiente para o treinamento de um modelo. A fase de treinamento, nesses casos, é feita em uma máquina de desenvolvimento, para que, depois, apenas a inferência seja realizada no dispositivo de borda \cite{soro2021tinyml}.

\begin{sloppypar}
Para o desenvolvimento de TinyML em microcontroladores, há diversos \textit{frameworks} disponíveis, incluindo TFLM\footnote{Documentação do TFLM: \url{https://github.com/tensorflow/tflite-micro}} (Google), uTensor \footnote{Documentação do uTensor: \url{https://github.com/uTensor/uTensor}} (ARM), MicroTVM\footnote{Documentação do MicroTVM: \url{https://daobook.github.io/tvm/docs/topic/microtvm/index.html}} (uTVM), STM32Cube.AI\footnote{Documentação do STM32Cube.AI: \url{https://stm32ai.st.com/stm32-cube-ai/}} (STMicroelectronics), NanoEdge AI Studio\footnote{Site do NanoEdge AI Studio: \url{https://stm32ai.st.com/nanoedge-ai/}} (STMicroelectronics), Eloquent TinyML\footnote{Documentação do Eloquent TinyML: \url{https://github.com/eloquentarduino/EloquentTinyML}}, emlearn\footnote{Documentação do emlearn: \url{https://github.com/emlearn/emlearn}} e EON compiler\footnote{Documentação do EON Compiler: \url{https://docs.edgeimpulse.com/docs/edge-impulse-studio/deployment/eon-compiler}} (Edge Impulse). Embora cada um apresente suas particularidades e suporte a diferentes famílias de MCUs, o TFLM foi selecionado para este trabalho por suas vantagens e conveniência, alinhando-se aos interesses do projeto. 
\end{sloppypar}

Dentre as conveniências estão o suporte para microcontroladores da família STM32 (baseada na arquitetura ARM Cortex-M) tanto pelo Robcmp quanto pelo TFLM, a existência de modelos já treinados em formato .tflite que podem ser aproveitados, e uma licença que permite uso gratuito, modificação e distribuição, a qual pode ser encontrada no repositório do GitHub, que aponta para a licença Apache License, Version 2.0\footnote{Disponível em: \url{https://www.apache.org/licenses/LICENSE-2.0}}. Dentre os interesses do projeto estão: manter o código aberto, buscar expandir para outras plataformas futuramente e o foco em portabilidade e suporte a \textit{hardware} heterogêneo.

\subsection{TensorFlow Lite Micro}

Conforme \citeonline{MLSYS2021_6c44dc73}, o TFLM possui uma série de vantagens que podem ser observadas após a análise das decisões de design e implementação. Dentre essas vantagens, as mais pertinentes e em conformidade com o objetivo da RL são:

\begin{itemize}
    \item Código Aberto: TFLM é um \textit{framework} de inferência TinyML de código aberto disponível no GitHub.
    \item Abordagem Baseada em Interpretador: Utiliza um interpretador que oferece flexibilidade e portabilidade, facilitando a adaptação a novas aplicações e recursos. Diferente da geração de código, o interpretador permite atualizar modelos substituindo apenas o arquivo/área de memória do modelo, sem recompilar tudo.
    \item Independência de \textit{hardware}: Minimiza o uso de dependências externas e requisitos de biblioteca para ser independente em relação ao \textit{hardware}.
    \item Gerenciamento de Memória Eficiente: Não depende de alocação dinâmica; utiliza uma ``arena'' de memória fornecida pela aplicação.
    \item Reutilização de Ferramentas TensorFlow: Integra-se fortemente com o ambiente de treinamento do TensorFlow e reutiliza as ferramentas e \textit{kernels} de referência do TensorFlow Lite, facilitando a conversão, otimização e garantindo um ambiente harmonizado.
    \item Suporte a Otimizações: Suporta quantização (como 8 bits) e outras otimizações (\textit{folding}, remoção de \textit{dropout}) através da \textit{toolchain} do TensorFlow Lite.
\end{itemize}

\section{Interoperabilidade entre linguagens de programação}

Conforme \citeonline{ISO_IEC_IEEE_24765_2017}, interoperabilidade pode ser definida como ``O grau em que dois ou mais sistemas, produtos ou componentes podem trocar informações e usar tais informações''. Em essência, a interoperabilidade é a capacidade de diferentes sistemas ou componentes de \textit{software} trocarem informações e utilizarem as funcionalidades uns dos outros, mesmo que tenham sido desenvolvidos com tecnologias, plataformas ou linguagens de programação distintas.

A interoperabilidade pode ser alcançada de diferentes maneiras, dependendo das tecnologias ou linguagens de programação utilizadas no projeto. Ela pode ocorrer entre diferentes linguagens de programação, por exemplo, C e Python, por meio de módulos existentes, como ctypes\footnote{Disponível em: \url{https://docs.python.org/3/library/ctypes.html}} e cffi\footnote{Disponível em: \url{https://cffi.readthedocs.io/en/latest/}}. Nesses casos, a interoperabilidade pode ser interpretada como Interface de Função Estrangeira (FFI, do inglês \textit{Foreign Function Interface}).

No contexto deste projeto, a RL será integrada ao TFLM. Para que a RL invoque as funções de inferência disponibilizadas pela API em C++, será criada uma API intermediária (\textit{wrapper}) implementada em C++, mas exposta por meio de uma interface em C. Isso permitirá a interoperabilidade entre os dois componentes em tempo de ligação (\textit{link-time}); assim, o código em RL, compilado com o robcmp, poderá ser ligado ao código C++ do TFLM, compilado com um compilador padrão C++.

\section{Robotics Language and Compiler}

Desenvolvidos na UFJ em 2018, os trabalhos de \citeonline{AnaliseRobEducacional} e \citeonline{RobEducacional} integraram o projeto ``Especificação e Construção de Protótipos Funcionais de Kits Robóticos de Baixo Custo para uso em Processos de Ensino-Aprendizagem'' (PI02361-2018), que visava criar um ecossistema de robótica educacional de baixo custo. Sousa dedicou-se à especificação do \textit{hardware}, avaliando componentes por preço e desempenho, enquanto Rodrigues criou o \textit{software}, desenvolvendo a Linguagem para Robótica Educacional (LRE) em português e seu respectivo compilador para facilitar a programação. Posteriormente, \citeonline{Majestic} realizou adaptações e melhorias na linguagem, como a mudança da língua para o inglês e a adição de funcionalidades como funções com parâmetros, vetores e matrizes estáticas.

Posteriormente, surgiu o projeto denominado ``Robotics Language: Uma Linguagem de Programação de Propósito Específico para Microcontroladores'' (PI05974-2024). Trabalhos relacionados a esse projeto implementam funcionalidades até então inexistentes na biblioteca padrão da linguagem. Por exemplo, \citeonline{Ryan} implementou funções matemáticas, como seno e cosseno. Já \citeonline{Gabriel} adicionou funções para manipulação de \textit{strings}.

A RL busca isolar as especificidades de microcontroladores, fornecendo uma camada de abstração de \textit{hardware} dentro do próprio compilador, em vez de depender de \textit{frameworks} ou bibliotecas externas. Essa abordagem permite que os desenvolvedores escrevam o código uma única vez, eliminando a necessidade de adaptar o código para cada \textit{hardware} através de macros condicionais. Ao aproveitar a análise semântica do compilador, é possível prevenir erros comuns encontrados no desenvolvimento
de \textit{firmware} em linguagens não específicas de domínio, como C/C++ \cite{borges_robcmp}. Além disso, ela possui suporte a interfaces, tipos complexos (com variáveis e métodos, similar a classes da orientação a objetos), além de um sistema de tipagem implícito.

Seu compilador, o Robcmp, foi escrito em C++ e utilizou ferramentas como o Flex\footnote{Disponível em: \url{https://github.com/westes/flex}} (versão 2.6.4) para análise léxica e o Bison\footnote{Disponível em: \url{https://www.gnu.org/software/bison}} (versão 3.8.2) para análise sintática. Para a interpretação da representação intermediária, foi utilizado o \textit{backend} oferecido pelo LLVM. Desde o começo do projeto, a versão do LLVM utilizada é constantemente atualizada, para que o compilador se mantenha sempre atualizado.

Para o desenvolvimento de programas e \textit{firmwares} na Robotics Language é recomendado o uso do editor de texto e IDE (\textit{Integrated Development Environment}) Visual Studio Code\footnote{Disponível em: \url{https://code.visualstudio.com}}, devido ao seu fácil acesso a extensões. A principal extensão necessária para o desenvolvimento é a PlatformIO\footnote{Disponível em: \url{https://platformio.org/}}. Ela é responsável pela execução, depuração e carregamento dos \textit{firmwares}. Outra extensão essencial é a RobCmpSyntax, criada com o intuito de ajudar o programador a visualizar a estrutura de seu código a partir de cores diferentes para a sintaxe.

\section{Lacuna e Justificativa}
Seção sugerida pela Ariadne.

% ---------------------------------------------------------------------------
% \section{Considerações Finais}
% Tanto a seção Introdução como a seção Considerações Finais são opcionais na construção de um texto monográfico. Porém deve-se optar por um padrão. Coloca-se em todos os capítulos ou em nenhum...

% \section{Formatação da monografia}
% \indent Os textos devem ser apresentados em papel branco, formato A4. Deverá ser digitado em tinta cor preta, com exceção de ilustrações. As folhas deverão apresentar margem esquerda e superior de 3 cm; direita e inferior de 2 cm. De forma geral este modelo (template) apresenta todas formatações necessárias. 
% A tabela \ref{tab1} é um exemplo. 


% \begin{table}[htp]
% \caption{Modelo de Tabela com referência \cite{hancock1995virtual}}  
%  \begin{center}
%   \begin{tabular}{c|c|c|c}
%    \hline
%    Características & C1 & C2 & C3 \\
%    \hline
%    1000 & 2000 & 3000 & 1000 \\
%    4000 & 2000 & 3000 & 1000 \\
%    5000 & 3000 & 1000 & 1000 \\
%    \hline
%   \end{tabular}
%   \label{tab1}
%  \end{center}
% \end{table}

 
% Tabela \ref{tab1} mostra .....

% \begin{figure}[htp]
% \centering
% \includegraphics[scale=0.9]{img/ModeloImagem.jpg}
% \caption{Modelo de Imagem \cite{pimentel1995}}
% \label{modeloimg}
% \end{figure}

 



\chapter{Trabalhos relacionados}\label{cap:relacionados}

\section{Introdução}

Neste capítulo, são apresentadas publicações relacionadas a este trabalho, que abordam temas como, decisões de design de ferramentas para \textit{TinyML} e \textit{Edge AI}.

\section{Critérios de busca}

\section{Metodologia de análise}

\subsection{Critério 1 (C1)}
\subsection{Critério 2 (C2)}
\subsection{Critério 3 (C3)}
\subsection{Critério 4 (C4)}

\section{Trabalhos analisados}
Com base nos critérios apresentados na seção anterior, foram analisados X trabalhos...

\subsection{Trabalho 1 (T1)}
\subsection{Trabalho 2 (T2)}
\subsection{Trabalho 3 (T3)}

\section{Resumo Comparativo}

Observa-se que...

Como verificado na Tabela \ref{comparativo}, os trabalhos T1 e T2 utilizam ...
\begin{table}[h]
\centering 
\caption{Comparativo entre trabalhos}
\label{comparativo}
    \begin{tabular}{cccccc}
    ~  & C1  & C2  & C3 $\alpha$  & C3 $\beta$ & C4  \\ \hline
    T1 			& Sim & Não & Sim & Não & Sim 	\\ \hline
    T2 			& Não & Não & Nao & Não & Sim	\\ \hline
    T3 			& Sim & Sim & Sim & Não & Sim 	\\ \hline
    TR			& Sim & Sim & Sim & Sim & Sim 	\\ \hline
    
    \end{tabular}
\end{table}

Dessa forma, os trabalhos elencados neste capitulo ...




\chapter{Metodologia}\label{cap:arquitetura}

\section{Introdução}
Este capítulo descreve os procedimentos metodológicos adotados e os recursos empregados para alcançar os objetivos deste trabalho. A apresentação é dividida em seções que abordam, respectivamente, a classificação formal da pesquisa, os materiais de software e hardware utilizados e, por fim, o projeto e a arquitetura da solução que serviram de base para a implementação detalhada no próximo capitulo.

\section{Classificação da pesquisa}
Este projeto propôs ampliar a linguagem de programação RL, adicionando novas funcionalidades de TinyML por meio da integração de uma biblioteca de software. Sendo assim, quanto à sua natureza, esta pesquisa caracterizou-se como aplicada, pois buscou gerar conhecimento para uso prático direto na ampliação da linguagem RL com funcionalidades de TinyML. Quanto aos objetivos, foi exploratória, pois investigou como tais recursos poderiam ser incorporados na linguagem. Em relação aos procedimentos, além de ter sido documental e bibliográfica, por se basear em artigos e materiais já publicados, a pesquisa foi também experimental, uma vez que envolveu a implementação prática e a avaliação das novas funcionalidades na linguagem RL. Por fim, a abordagem foi quantitativa, ao avaliar o desempenho por meio de dados como latência, e qualitativa, com a análise de atributos como simplicidade e manutenibilidade da solução.

\section{Materiais}
Nessa seção, serão abordados todos os materiais que serão utilizados durante a pesquisa. Isso inclui não somente dispositivos físicos, mas também as tecnologias e os \textit{softwares} que serão empregados.

\subsection{Software}
Os novos recursos e funcionalidades implementados diretamente na linguagem RL foram escritos em C++ e desenvolvidos com o auxílio das ferramentas Flex (versão 2.6.4) e Bison (versão 3.8.2), responsáveis por gerar os analisadores léxico e sintático do Robcmp. As adições foram incorporadas à sintaxe da linguagem por meio da criação de novas palavras reservadas.

O ambiente de desenvolvimento foi o Visual Studio Code (versão 1.101), com o auxílio de duas extensões principais: o PlatformIO (versão 6.1.18), para depurar e exportar os \textit{firmwares} para as placas, e o RobCmpSyntax (versão 1.0), que fornece o realce de sintaxe para arquivos .rob. Para a fase de testes e simulações, foi empregado o emulador QEMU\footnote{Disponível em: \url{https://www.qemu.org}} (versão 10.0.2), capaz de emular microcontroladores de famílias como STM32, que são baseados na arquitetura ARM Cortex-M e utilizados no projeto.

\subsection{Hardware}
A máquina de desenvolvimento utilizada foi um \textit{notebook} Lenovo IdeaPad 3 15ITL6, equipado com processador Intel® Core™ i7-1165G7, 16 GB de RAM e sistema operacional Ubuntu 24.04.1 LTS. Além da máquina de desenvolvimento, a execução e validação das aplicações ocorreram em uma placa com o microcontrolador STM32F407VET6, da família STM32, popularmente chamada de Black STM32F407VET6. Essa placa conta com um processador ARM Cortex-M4 rodando a 168 MHz, 192 KB de RAM e 512 KB de memória Flash.

\section{Projeto e Arquitetura da Solução}
Para que a integração fosse desenvolvida corretamente, foi fundamental definir quais funções da biblioteca TFLM seriam utilizáveis no Robcmp. Essa tarefa também incluiu a determinação de quais funcionalidades seriam nativas da linguagem e quais pertenceriam à biblioteca padrão.

A implementação de uma funcionalidade de forma nativa implicou em adicionar novas palavras reservadas à sintaxe da linguagem, ou seja, modificar as análises léxica e sintática. Em contrapartida, as funcionalidades que não modificaram as análises léxica e sintática do Robcmp foram oferecidas por meio da biblioteca padrão. Escritas em arquivos com a extensão .rob, elas puderam ser reaproveitadas pelos desenvolvedores ao importá-las no início de seus códigos. A biblioteca do Robcmp já contemplava alguns módulos, como o math, que podia ser importado para a utilização de funções matemáticas.

O \hyperref[fig:nova_sintaxe]{Código~\ref{fig:nova_sintaxe}} exibe um trecho de código que exemplifica o fluxo de um programa utilizando as novas palavras-chave da linguagem RL. A palavra-chave \texttt{model} foi utilizada para carregar o modelo e criar um objeto que conteria todas as informações necessárias para seu correto funcionamento. Já o comando \texttt{invoke} ficou responsável por realizar a inferência, ou seja, processar os dados de entrada (\textit{input}) para gerar a saída (\textit{output}).

% \lstset{language=rob,morekeywords=[2]{portmode,mcu,digitalport}}
% \begin{lstlisting}[float=tp, caption=A hardware-agnostic LED blink example., label={lst:ledblink}]
% // an mcu implementation will be bond here
% mmcu = mcu();
% // the firmware needs a digital port
% led = digitalport();

% int16 main() {
% 	led.mode(portmode.output);
% 	loop {
% 		led.set(true);     // turn on the LED
% 		mmcu.wait_ms(500);
% 		led.set(false);    // turn off the LED
% 		mmcu.wait_ms(500);
% 	}
% }
% \end{lstlisting}

%// Uma implementacao de microcontrolador sera vinculada aqui.
%mmcu = mcu();

\lstset{language=rob,morekeywords=[1]{portmode, mcu ,digitalport, model, invoke}}
\begin{lstlisting}[caption={Exemplo de código com a nova sintaxe para inferência de um modelo tflite.}, label={fig:nova_sintaxe}]
int16 main() {
	// 1. Define os dados de entrada e um buffer para a saida.
	input = {0.77, 1.57, 2.3, 3.14};
	output = {5:0.0};

	// 2. Carrega o modelo
	model meu_modelo("meu_modelo.tflite", arena_size: 4096);

	// 3. Fornece a entrada para o modelo.
	meu_modelo.input = input;

	// 4. Executar a inferencia.
	invoke meu_modelo;
	
	// 5. Obtem o resultado.
	output = meu_modelo.output;
	//Resto do codigo
}
\end{lstlisting}

\section{Procedimentos de Validação Funcional}
A etapa de validação funcional teve como objetivo verificar se a solução proposta cumpria o que prometia, sem levar em consideração sua eficiência, aplicabilidade, facilidade ou qualquer outro fator que mede sua qualidade.

Nesta etapa, as novas funcionalidades foram validadas utilizando o simulador QEMU, com suporte para a emulação de plataformas da família STM32. Nesse processo, o QEMU foi empregado para executar e testar os firmwares gerados pelo compilador Robcmp, que utilizaram diferentes modelos de ML. O emulador operou no modo de Emulação de Sistema Completo (Full System Emulation), atuando como uma “placa de desenvolvimento virtual” que consegue executar o firmware e simular os periféricos necessários. A configuração e a operação da máquina virtual foram controladas por linha de comando, por meio de parâmetros que especificavam o modelo da plataforma, a CPU exata, o arquivo de firmware a ser executado, entre outros aspectos.

\section{Validação da eficiência e Análise Comparativa}
Para validar a eficiência da integração do TFLM à RL, foi realizada uma análise comparativa entre duas abordagens de desenvolvimento de firmwares. Foi comparada a solução proposta em RL com um firmware implementado diretamente em C++, utilizando o próprio TFLM como base para ambos. Para garantir uma avaliação justa, ambos os testes foram conduzidos em um MCU idêntico e com o mesmo modelo de ML. Nesta avaliação, foram considerados critérios como a facilidade de desenvolvimento, a manutenibilidade do código, a latência de inferência e o tamanho final do executável.









\chapter{Implementação}\label{cap:implementacao}

\section{Introdução}
Neste capítulo são apresentados os detalhes da implementação das funcionalidades propostas neste trabalho. O código-fonte completo pode ser encontrado no repositório online \url{https://github.com/LuizEduardoRezende/robcmp/tree/tflm-front-end}. O referido repositório é um \textit{fork} do projeto Robcmp original, e todo o desenvolvimento foi realizado em duas \textit{branchs} dedicadas, \texttt{tflm} e \texttt{tflm-front-end}. Futuramente, será submetido um \textit{pull request} para integrar estas contribuições à \textit{branch} principal do projeto Robcmp, a fim de que as novas funcionalidades fiquem disponíveis para a comunidade.

\section{Arquitetura da Camada de Interoperabilidade}
O processo de compilação e linkedição do Robcmp pode ser explicado da seguinte forma: programas escritos com a extensão \texttt{.rob} são compilados pelo Robcmp e geram arquivos objeto (\texttt{.o}). Já as bibliotecas externas, como o TFLM ou outras, são compiladas separadamente, resultando em arquivos de biblioteca estática (\texttt{.a}). Após a compilação de todos os programas e bibliotecas necessários, o \textit{linker} entra em ação com o objetivo de resolver todas as referências entre os componentes e criar um executável único, que pode ser chamado de \textit{firmware}.

É válido ressaltar que a biblioteca estática do TFLM pode ser facilmente compilada para diferentes arquiteturas, utilizando o sistema de \texttt{Makefile} do proprio \textit{framework}. Diferente do arquivo objeto do programa em RL, que é gerado pelo próprio compilador Robcmp. Por fim, o \textit{wrapper} C é compilado por meio do sistema de \texttt{CMakeLists} do Robcmp, que resolve todas as dependências e gera a biblioteca estática correspondente. A Figura~\ref{fig:linker} ilustra a arquitetura de compilação e interoperabilidade entre os componentes.

\begin{figure}[H]
\centering
\includegraphics[width=0.6\linewidth]{Imagens/linker.png}
\caption{Ilustração da arquitetura de compilação e interoperabilidade. As bibliotecas estáticas (\texttt{.a}) do TFLM e do \textit{wrapper} C, assim como o arquivo objeto (\texttt{.o}) do programa em RL, são processados de forma independente e, na etapa final, unificados pelo \textit{linker}. O \textit{linker} resolve as referências entre os componentes para criar o executável único.}
\label{fig:linker}
\end{figure}


\section{Implementação do \textit{wrapper} C}
O principal objetivo do \textit{wrapper} C é fornecer uma interface simples e direta para que o código gerado pela RL possa interagir com a API do TFLM. Embora o arquivo-fonte tenha a extensão \texttt{.cpp} para permitir a chamada direta à API C++ do TFLM, todas as funções de interface foram declaradas com \texttt{extern “C”} para garantir uma ligação C (\textit{C linkage}) pura. Essa abordagem permite que o código compilado da RL se conecte ao \textit{wrapper} como se ele fosse uma biblioteca C nativa, resolvendo a interoperabilidade em tempo de linkagem. O código do \textit{wrapper} pode ser encontrado no repositório do Robcmp, na pasta \texttt{/wrappers} ou no link direto \url{https://github.com/LuizEduardoRezende/robcmp/blob/tflm-front-end/wrappers/tflm/tflm_wrapper.cpp}.

\lstset{
  language=C++,
  basicstyle=\ttfamily\small,
  keywordstyle=\color{red}\bfseries,
  commentstyle=\color{gray}\itshape,
  stringstyle=\color{orange},
  numbers=left,
  numberstyle=\tiny,
  breaklines=true,
  emph={TFLM_Instance, RegisterOp, InitializeInterpreter, GetOutputTensor, GetInputTensor, SetTensorValue, GetTensorAsFloat, SetTensorArray, GetTensorArray, InvokeInterpreter}, emphstyle=\color{blue}\bfseries,
}

\subsection{Estrutura \texttt{TFLM\_Instance}}
A estrutura \texttt{TFLM\_Instance} encapsula todos os objetos principais do TFLM, o interpretador, o alocador de memória e o resolvedor de operações. Isso mantém o estado de cada modelo carregado de forma organizada.

\begin{lstlisting}[language=C++, caption={Struct \texttt{TFLM\_Instance} encontrada no começo do \textit{wrapper}}, label={lst:tflm-instance}]
struct TFLM_Instance {
    tflite::RecordingMicroInterpreter* interpreter;
    tflite::RecordingMicroAllocator* allocator;
    MutableResolver* resolver;
};
\end{lstlisting}

\subsection{Tipo enumerado \texttt{KernelType} e função \texttt{RegisterOp}}
A enumeração \texttt{KernelType}, inspirada na enumeração \texttt{BuiltinOperator} do próprio TFLM, foi criada com o intuito de possibilitar o registro dinâmico de operadores (\textit{kernels}) necessários para a execução de um modelo. Tanto a enumeração quanto a função \texttt{RegisterOp} atuam em conjunto para garantir que apenas os kernels essenciais, como \texttt{CONV\_2D} ou \texttt{FULLY\_CONNECTED}, sejam carregados dinamicamente conforme a necessidade especificada pelo desenvolvedor.

\begin{lstlisting}[language=C++, caption={Tipo enumerado \texttt{KernelType} e função \texttt{RegisterOp}}, label={lst:kernel-type}]
typedef enum {
  ADD = 0,
  AVERAGE_POOL_2D = 1,
  CONCATENATION = 2,
  CONV_2D = 3,
  DEPTHWISE_CONV_2D = 4,
  /* ... demais operadores ... */
} KernelType;

void RegisterOp(tflite::MicroMutableOpResolver<100>* resolver, KernelType kernel_type) {
    switch (kernel_type) {
        case ADD: resolver->AddAdd(); break;
        case AVERAGE_POOL_2D: resolver->AddAveragePool2D(); break;
        case CONCATENATION: resolver->AddConcatenation(); break;
        case CONV_2D: resolver->AddConv2D(); break;
        /* ... demais casos ... */
        default:
            MicroPrintf("AVISO: Tipo de kernel nao mapeado: %d", kernel_type);
            break;
    }
}

\end{lstlisting}

\subsection{\texttt{InitializeInterpreter}}
É a primeira função a ser chamada em um fluxo de inferência, responsável por configurar o interpretador e outras instâncias necessárias para a execução do modelo. Recebe como parâmetros os dados do modelo em formato de array de bytes, o buffer de memória pré‑alocado (tensor arena) e a lista de kernels a serem registrados. Retorna um identificador único (handle) para a instância do modelo, do tipo \texttt{uintptr\_t}, pois o Robcmp não possui suporte a ponteiros nativos em sua linguagem.

É importante ressaltar que a função recebe alguns parâmetros adicionais além dos já mencionados, como \texttt{tensor\_arena\_size}, \texttt{num\_kernels} e um parâmetro sem nome. Esses parâmetros podem ou não ser utilizados, mas devem constar na declaração da função por questões de compatibilidade com o \textit{backend} do LLVM. As demais funções do wrapper também podem conter parâmetros extras pelo mesmo motivo.

A necessidade desses parâmetros surge quando o Robcmp utiliza a biblioteca padrão do diretório \texttt{lib/ai/tflm.rob}, que contém as declarações das funções do wrapper. Essas declarações precisam estar alinhadas com as definições reais das funções no wrapper. No arquivo \texttt{.rob}, para cada argumento que seja um vetor, é adicionado um argumento extra do tipo inteiro que representa o tamanho desse vetor. Isso é uma característica do compilador Robcmp para tratar vetores em chamadas de função. Portanto, para manter a consistência entre declarações e definições, esses parâmetros adicionais são incluídos nas definições das funções no wrapper.

No caso de uso do wrapper sem a biblioteca padrão \texttt{tflm.rob}, ou seja, quando as chamadas são geradas diretamente pelo front-end do compilador, esses parâmetros adicionais não são necessários, pois as chamadas ocorrem diretamente no código C++ ou no LLVM-IR gerado pela análise semântica do compilador.

\begin{lstlisting}[language=C++, caption={Assinatura da função \texttt{InitializeInterpreter}}, label={lst:initialize-interpreter}]
uintptr_t InitializeInterpreter(const uint8_t* model_data,
                               uint8_t* tensor_arena,
                               const uint8_t* required_kernels,
                               int,
                               int tensor_arena_size,
                               int8_t num_kernels);
\end{lstlisting}

\subsection{\texttt{GetInputTensor} e \texttt{GetOutputTensor}}
Funções simples que retornam ponteiros para os tensores de entrada e saída do modelo, respectivamente. Esses ponteiros são utilizados para leitura e escrita dos dados posteriormente. As funções recebem como parâmetro o \textit{handle} do modelo e também o index do tensor desejado. Modelos podem ter mais de um tensor de entrada ou saída, ou seja , um modelo pode retornar diferentes arrays de saída dependendo da sua arquitetura.

\begin{lstlisting}[language=C++, caption={Assinatura das funções \texttt{GetInputTensor} e \texttt{GetOutputTensor}}, label={lst:get-tensor}]
uintptr_t GetOutputTensor(uintptr_t instance_handle, size_t index, int);

uintptr_t GetInputTensor(uintptr_t instance_handle, size_t index, int);
\end{lstlisting}


\subsection{Convenção de Tipos de Dados e Quantização}
Modelos de TinyML frequentemente usam tipos de dados otimizados, como inteiros de 8 bits, para economizar memória e acelerar o processamento. O \textit{wrapper} simplifica isso para o usuário da RL. As funções do código \hyperref{lst:get-tensor-value} convertem automaticamente valores de ponto flutuante para o formato quantizado exigido pelo modelo e vice-versa. O desenvolvedor em RL pode trabalhar sempre com float, e o \textit{wrapper} cuida de toda a matemática de conversão internamente. 

Enquanto que as funções \texttt{SetTensorValue} e \texttt{GetTensorAsFloat} são para manipular valores individuais, as funções \texttt{SetTensorArray} e \texttt{GetTensorArray} permitem manipular arrays inteiros dos tensores.

\begin{lstlisting}[language=C++, caption={Assinatura das funções de resgate e passagem de valores.}, label={lst:get-tensor-value}]
void SetTensorValue(uintptr_t tensor_handle, size_t index, float value, int);

float GetTensorAsFloat(uintptr_t tensor_handle, size_t index, int);

void SetTensorArray(uintptr_t tensor_handle, const float* values, size_t count, int);

void GetTensorArray(uintptr_t tensor_handle, float* values, size_t max_count, int);
\end{lstlisting}

\subsection{\texttt{InvokeInterpreter}}
Essa é uma função responsável pela inferência do modelo. Ela aciona o TFLM para executar o modelo com os dados de entrada fornecidos. É o “cérebro” da operação, funcionando como um botão de ação.

\begin{lstlisting}[language=C++, caption={Assinatura da função \texttt{InvokeInterpreter}}, label={lst:invoke}]
void InvokeInterpreter(uintptr_t instance_handle, int);
\end{lstlisting}

\subsection{Fluxo de Trabalho de Inferência}
Para realizar uma inferência com um modelo TinyML utilizando o \textit{wrapper}, o fluxo de trabalho segue os seguintes passos principais, por meio das chamadas de função correspondentes:

\begin{enumerate}
    \item Inicialização (\texttt{InitializeInterpreter}): Esta é a etapa inicial do processo, onde são passados os dados necessários para configurar o modelo e é recebido um \textit{handle} que representa a instância do modelo carregado.
        
    \item Acesso aos tensores (\texttt{GetInputTensor} e \texttt{GetOutputTensor}): Com o \textit{handle} do modelo, é possível obter ponteiros para os tensores de entrada e saída, que são utilizados para leitura e escrita dos dados.
        
    \item Fornecimento dos dados de entrada (\texttt{SetTensorValue}) e (\texttt{SetTensorArray}): A partir dos handles dos tensores, os dados de entrada são inicializados.
        
    \item Execução da inferência (\texttt{InvokeInterpreter}): Esta função é acionada para que os dados de entrada possam ser processados pelo modelo, gerando os resultados no tensor de saída.
        
    \item Obtenção dos resultados (\texttt{GetTensorAsFloat} e \texttt{GetTensorArray}): Após a inferência, essas funções são usadas para ler os resultados do tensor de saída. A função retorna os dados em ponto flutuante da mesma forma padronizada que os dados de entrada foram passados.

    \item Liberação de recursos (\texttt{DestroyInterpreter}): Essa função pode ser chamada para liberar os recursos para a instância do modelo, evitando vazamentos de memória.
\end{enumerate}

\section{Adaptação do \textit{frontend} do Compilador}

Toda implementação do \textit{frontend} está localizada na pasta \texttt{src/}, portanto, todos arquivos mencionados nesta seção podem ser encontrados nesse diretório do repositório do Robcmp.

\lstdefinelanguage{Flex}{
  morekeywords={return},
  sensitive=true,
  morecomment=[l]{//},
  morecomment=[s]{/*}{*/},
  morestring=[b]",
  alsoletter={_},
  emph={TOK_MODEL, TOK_INVOKE, TOK_MODEL_INPUT, TOK_MODEL_OUTPUT}, emphstyle=\color{blue}\bfseries,
}
\lstset{
  language=Flex,
  basicstyle=\ttfamily\small,
  keywordstyle=\color{red}\bfseries,
  commentstyle=\color{gray}\itshape,
  stringstyle=\color{orange},
  numbers=left,
  numberstyle=\tiny,
  breaklines=true,
  tabsize=2,
  showstringspaces=false
}

\subsection{Análise léxica}
O arquivo Flex responsável pela análise léxica do Robcmp é o \texttt{Language.l}, cujo objetivo é identificar os tokens da linguagem RL. Foram adicionados quatro novos tokens para suportar as palavras‑chave e as funcionalidades: \texttt{TOK\_MODEL}, \texttt{TOK\_INVOKE}, \texttt{TOK\_MODEL\_INPUT} e \texttt{TOK\_MODEL\_OUTPUT}.

Enquanto as expressões regulares para os tokens \texttt{TOK\_MODEL} e \texttt{TOK\_INVOKE} são simples, correspondendo diretamente às palavras‑chave ``model'' e ``invoke'', as expressões para \texttt{TOK\_MODEL\_INPUT} e \texttt{TOK\_MODEL\_OUTPUT} são mais complexas: identificam padrões formados por um identificador seguido de ``.input'' ou ``.output'' e capturam o lexema completo para uso posterior nas análises sintática e semântica.


\begin{lstlisting}[language=Flex,caption={Novos \textit{tokens} do \textit{Scanner} Flex},label={lst:language-l}]
"model"					{ return TOK_MODEL; }  // TFLM
"invoke"				{ return TOK_INVOKE; } // TFLM

{ID}\.input			{ yylval->ident = strndup(yytext, yyleng);
						  return TOK_MODEL_INPUT;
						}

{ID}\.output			{ yylval->ident = strndup(yytext, yyleng);
						  return TOK_MODEL_OUTPUT;
						}
\end{lstlisting}

\lstdefinelanguage{Bison}{
sensitive=true,
morekeywords={\%token,\%left,\%right,\%nonassoc,\%union,\%type,\%start,\%prec, |},
morecomment=[l]{//},
morecomment=[s]{/*}{*/},
morestring=[b]",
alsoletter={_\%} 
}
\lstset{
language=Bison,
basicstyle=\ttfamily\small,
keywordstyle=\color{red}\bfseries,
commentstyle=\color{gray}\itshape,
stringstyle=\color{orange},
numbers=left,
numberstyle=\tiny,
breaklines=true,
showstringspaces=false,
emph={},
literate={\$}{{\$}}1 {@}{{@}}1 {\{}{{\{}}1 {\}}{{\}}}1
}

\subsection{Análise sintática}
No Robcmp, os arquivos responsáveis pela análise sintática são \texttt{Language.y}, \texttt{LanguageHeader.y} e \texttt{LanguageUse.y}. O primeiro define a gramática principal da linguagem RL, enquanto os dois últimos tratam, respectivamente, das declarações de cabeçalho e das regras sintáticas para a inclusão da biblioteca padrão.

Este trabalho concentrou-se principalmente no arquivo \texttt{Language.y}, onde foram adicionadas novas regras sintáticas à gramática principal da linguagem, e também no arquivo \texttt{LanguageHeader.y}, para suportar os novos tokens e novos tipos.

Inicialmente, antes de criar as novas regras sintáticas, foi necessário adicionar os novos tokens ao arquivo \texttt{LanguageHeader.y}, conforme mostrado no código \hyperref[lst:language-header]{\ref{lst:language-header}}. Além disso, foi preciso definir um novo tipo de dado para representar os modelos TinyML. Esse tipo foi adicionado ao arquivo \texttt{Header.y}, responsável por definir todos os tipos possíveis da AST de um programa, conforme mostrado no código \hyperref[lst:language-types]{\ref{lst:language-types}}.


\begin{lstlisting}[language=Bison,caption={Novos tokens adicionados no \texttt{LanguageHeader.y}},label={lst:language-header}]
%token TOK_MODEL TOK_INVOKE TOK_MODEL_INPUT TOK_MODEL_OUTPUT
\end{lstlisting}

\begin{lstlisting}[language=C++,caption={Novo tipo de dado definido no arquivo \texttt{Header.h}},label={lst:language-types}]
#include "ModelNode.h"
\end{lstlisting}

Após a fase de preparação, as novas regras sintáticas são definidas, juntamente com os novos tokens e seus respectivos tipos de dados no arquivo \texttt{Language.y}. Conforme mostrado no código \hyperref[lst:language-rules]{\ref{lst:language-rules}}, primeiramente os tokens são declarados e seus tipos de dado são definidos. Em seguida, as novas regras são implementadas na posição apropriada da gramática. A regra de declaração de um modelo (\texttt{model\_stmt}) é adicionada às produções \texttt{global} e \texttt{stmt}, permitindo que modelos sejam declarados tanto no escopo global quanto dentro de funções ou blocos. Por outro lado, a regra de invocação de um modelo (\texttt{invoke\_stmt}) é adicionada apenas à produção \texttt{stmt}, já que invocações de modelos não fazem sentido no escopo global. Além disso, as regras para manipulação de tensores são incorporadas nas produções \texttt{factor} e \texttt{complexvar\_set}, permitindo que tensores de entrada e saída sejam tratados como variáveis complexas e como expressões, respectivamente.

Depois que o compilador reconhece uma sequencia de tokens que se encaixa nessas novas regras sintáticas, ele cria os nós correspondentes na AST. Esses nós são instâncias da classe \texttt{ModelNode}, que encapsula todas as informações relevantes sobre o modelo TinyML, como o nome do modelo, o tipo de operação (entrada, saída ou invocação), os parâmetros associados e a localização no código-fonte para fins de depuração.

\begin{lstlisting}[language=Bison,caption={Declaração dos tokens e das regras sintáticas no \texttt{Language.y}},label={lst:language-rules}, escapeinside={(*@}{@*)}, literate={\$}{{\$}}1 {@}{{@}}1, columns=flexible, keepspaces=true]
%type <node> model_stmt invoke_stmt
%type <ident> TOK_MODEL_INPUT TOK_MODEL_OUTPUT

global : use
       | function
       /* ... demais casos ... */
       | model_stmt ';'  // TFLM

stmt :
     /* ... demais casos ... */
     | model_stmt ';'
     | invoke_stmt ';'

factor :
       /* ... demais casos ... */
       | TOK_MODEL_OUTPUT[id] '[' expr ']' { 
         // Extract model name from "model.output"
         std::string fullName($id);
         std::string modelName = fullName.substr(0, fullName.find('.'));
         $$ = new ModelNode(modelName.c_str(), "output", $expr, nullptr, @id);
         $$->setLocation(@id); 
       }

model_stmt : TOK_MODEL TOK_IDENTIFIER '(' paramscall ')' (*@ \{ @*)
    $$ = new ModelNode($2, $4, @2);
(*@ \} @*)

invoke_stmt : TOK_INVOKE TOK_IDENTIFIER (*@ \{ @*)
    $$ = new ModelNode($2, "invoke", @2);
(*@ \} @*)

complexvar_set : TOK_MODEL_INPUT[id] '[' expr ']' '=' logicexpr { 
	// Extract model name from "model.input"
	std::string fullName($id);
	std::string modelName = fullName.substr(0, fullName.find('.'));
	$$ = new ModelNode(modelName.c_str(), "input", $expr, $logicexpr, @id);
	$$->setLocation(@id); 
}
\end{lstlisting}


\subsection{Análise semântica}
A análise semântica no Robcmp é realizada por diversos arquivos C++ que são responsáveis por verificar as regras semânticas para diferentes tipos de nós na AST. Para criar as regras semânticas relacionadas aos modelos TinyML, foram criados os arquivos \texttt{ModelNode.h} e \texttt{ModelNode.cpp}, que definem a classe \texttt{ModelNode}. Essa classe herda da classe base \texttt{Node} e implementa os métodos necessários para a verificação semântica, geração de código LLVM-IR e outras funcionalidades.

O método mais importante da classe \texttt{ModelNode} é o \texttt{generate}, este método funciona como um visitor que é chamado pelo compilador ao encontrar um nó deste tipo na AST enquanto percorre a árvore. Dependendo das características do nó que chama o \texttt{generate}, ele gera o código LLVM-IR correspondente para a declaração do modelo, invocação do modelo ou manipulação dos tensores de entrada e saída.

O comportamento da análise semântica se divide em duas operações principais: o processamento da declaração de um modelo (\texttt{model\_stmt}) e o processamento do acesso aos seus membros (\texttt{.input}, \texttt{.output}, \texttt{invoke}). O método \texttt{generate} inicial é visto no código \hyperref[lst:modelnode-generate]{\ref{lst:modelnode-generate}}, onde é decidido qual operação será realizada com base na existência dos \texttt{params}, que indicam uma declaração de modelo, ou na ausência deles, que indica uma invocação do modelo.

\begin{lstlisting}[language=C++, caption={Método \texttt{generate} da classe \texttt{ModelNode}}, escapeinside={(*@}{@*)}, label={lst:modelnode-generate}]
Value* ModelNode::generate(FunctionImpl *func, BasicBlock *block, BasicBlock *allocblock) {
    if (!getScope()) (*@ \{ @*)
        setScope(func);
    (*@ \} @*)
    
    if (params) (*@ \{ @*)
        return generateDeclaration(func, block, allocblock);
    (*@ \} @*) else if (!memberName.empty()) (*@ \{ @*)
        return generateMemberAccess(func, block, allocblock);
    (*@ \} @*)
    return nullptr;
(*@ \} @*)
\end{lstlisting}


\subsection{Processamento da Declaração do Modelo (\texttt{model\_stmt})}
Caso o nó visitado pelo compilador seja um ModelNode com a presença de parâmetros, o método \texttt{generateDeclaration} é chamado, conforme mostrado no código \hyperref[lst:modelnode-generate]{\ref{lst:modelnode-generate}}. Esse método é responsável por gerar o código LLVM-IR necessário para declarar e inicializar uma instância do modelo TinyML. Ele realiza as seguintes etapas principais:

\begin{enumerate}

  \item Validação do Arquivo de Modelo: o compilador verifica se o arquivo do modelo especificado nos parâmetros existe e é acessível. Caso contrário, um erro semântico é gerado.
  \item Validação dos outros Parâmetros: o compilador verifica se os parâmetros \texttt{arena\_size} e \texttt{kernels} existem e são válidos. Caso contrário, um erro semântico é gerado.
  \item Geração de código LLVM-IR: cria várias variáveis globais que armazenam os dados do modelo, tamanho dos dados do modelo, \texttt{tensor\_arena}, instância do interpretador e o array de kernels.
  \item Chamada da função do \textit{wrapper} C \texttt{InitializeInterpreter}: o compilador gera uma chamada para a função \texttt{InitializeInterpreter} do \textit{wrapper} C, passando os parâmetros necessários, e armazena o \textit{handle} retornado.
  \item Retorna o \textit{handle} do modelo: o método retorna o \textit{handle} do modelo, que será utilizado em chamadas subsequentes para manipular o modelo.

\end{enumerate}

\subsection{Processamento de Acesso a Membros (.input, .output, invoke)}
Caso o nó visitado pelo compilador possua um nome de membro, o método \texttt{generateMemberAccess} é chamado, conforme mostrado no código \hyperref[lst:modelnode-generate]{\ref{lst:modelnode-generate}}. Esse método é responsável por rotear o acesso aos membros depenendo do nome do membro acessado. 

Inicialmente o método verifica se o modelo foi declarado anteriormente, caso contrário, um erro semântico é gerado. Em seguida, dependendo do nome do membro acessado, o método chama uma das três funções específicas: \texttt{generateInputAccess}, \texttt{generateOutputAccess} ou \texttt{generateInvoke}. Caso o nome do membro não seja reconhecido, um erro semântico é gerado, conforme mostrado no código \hyperref[lst:modelnode-generate]{\ref{lst:modelnode-generate}}.


\begin{lstlisting}[language=C++, caption={Roteamento do método \texttt{generateMemberAccess}}, escapeinside={(*@}{@*)}, label={lst:modelnode-generate}]
    if (memberName == "input") (*@ \{ @*)
        return generateInputAccess(func, block, allocblock, modelInstance);
    (*@ \} @*) else if (memberName == "output") (*@ \{ @*)
        return generateOutputAccess(func, block, allocblock, modelInstance);
    (*@ \} @*) else if (memberName == "invoke") (*@ \{ @*)
        return generateInvoke(func, block, allocblock, modelInstance);
    (*@ \} @*) else (*@ \{ @*)
        yyerrorcpp("Membro '" + memberName + "' nao suportado no modelo '" + modelName + "'. Use 'input', 'output' ou 'invoke'.", this);
        setSemanticError();
        return nullptr;
    (*@ \} @*)
\end{lstlisting}

\subsection{\texttt{generateInputAccess}}

\subsection{\texttt{generateOutputAccess}}

\subsection{\texttt{generateInvoke}}

\section{Considerações Finais}
Este trabalho tem como objetivo inicial implementar o suporte para TinyML na arquitetura ARM Cortex-M, utilizada por diversas famílias de microcontroladores, como a STM32. Contudo, como o compilador Robcmp está em constante evolução e pode suportar novas arquiteturas no futuro, o mesmo poderá ser feito para o TinyML. Para que isso ocorra, as bibliotecas do TFLM e os \textit{wrappers} deverão ser recompilados para a plataforma-alvo específica e disponibilizados no repositório do Robcmp.


\include{capitulos/5FuncDemo}
\chapter{Avaliação e Testes}\label{cap:analise}

Este capítulo abordará tudo relacionado ao processo de avaliação e testes da solução proposta. Inicialmente, serão descritos os procedimentos metodológicos adotados, incluindo também quais foram os recursos utilizados na execução dos testes, como hardware, software e outras ferramentas relevantes. Em seguida, será detalhada a configuração do ambiente experimental, fornecendo informações sobre como replicar o ambiente utilizado para os testes. Por fim, serão apresentados os resultados obtidos durante os testes, acompanhados de uma análise crítica e uma discussão final sobre os resultados.

\section{Metodologia}

Esta seção descreve a metodologia adotada para avaliação e testes da solução, apresentando o ambiente de desenvolvimento (software e hardware), bem como os procedimentos de validação funcional e a análise comparativa de eficiência.

\subsection{Software e Hardware}
Os novos recursos e funcionalidades implementados diretamente na linguagem RL foram desenvolvidos com o auxílio das ferramentas Flex (versão 2.6.4) e Bison (versão 3.8.2), responsáveis por gerar os analisadores léxico e sintático do Robcmp. Além disso, a análise semântica foi escrita em C++ conforme a estrutura do compilador. Por outro lado, a nova biblioteca padrão foi desenvolvida com a própria linguagem RL, criando um arquivo \texttt{.rob} que continha todas as declarações das funções do \textit{wrapper} C do TFLM.

O ambiente de desenvolvimento foi o Visual Studio Code (versão 1.101), com o auxílio de duas extensões principais: o PlatformIO (versão 6.1.18), para depurar e exportar os \textit{firmwares} para as placas, e o RobCmpSyntax (versão 1.0), que fornece o realce de sintaxe para arquivos .rob.

A máquina de desenvolvimento utilizada foi um \textit{notebook} Lenovo IdeaPad 3 15ITL6, equipado com processador Intel® Core™ i7-1165G7, 16 GB de RAM e sistema operacional Ubuntu 24.04.1 LTS. Além da máquina de desenvolvimento, a execução e validação das aplicações ocorreram em uma placa com o microcontrolador STM32F407VET6, da família STM32, popularmente chamada de Black STM32F407VET6. Essa placa conta com um processador ARM Cortex-M4 rodando a 168 MHz, 192 KB de RAM e 512 KB de memória Flash.

\subsection{Procedimentos de Validação Funcional}
A etapa de validação funcional teve como objetivo verificar se a solução proposta cumpria o que prometia, sem levar em consideração sua eficiência, aplicabilidade, facilidade ou qualquer outro fator que mede sua qualidade.

Nesta etapa, as novas funcionalidades foram validadas utilizando testes unitários, que foram executados tanto na máquina de desenvolvimento quanto na placa com o MCU. Esses testes foram projetados para verificar se cada nova funcionalidade implementada no Robcmp estava operando conforme o esperado. Foram criados diversos arquivos \texttt{.rob} que utilizavam diferentes modelos de ML, abrangendo uma variedade de cenários e casos de uso. Os testes foram desenvolvidos utilizando tanto a linguagem RL com as novas palavras-chave quanto a biblioteca padrão.

\subsection{Validação da eficiência e Análise Comparativa}
Para validar a eficiência da integração do TFLM à RL, foi realizada uma análise comparativa entre três abordagens de desenvolvimento de firmwares: os desenvolvidos com a nova sintaxe da RL, os que utilizaram a biblioteca padrão do Robcmp para TinyML e os desenvolvidos diretamente em C++ utilizando o ambiente do TFLM.

Para garantir uma avaliação justa, ambos os testes foram conduzidos em um MCU idêntico e com o mesmo modelo de ML. Nesta avaliação, foram considerados critérios como a facilidade de desenvolvimento, a manutenibilidade do código, o tempo de execução e o tamanho final do executável.

\section{Configuração do Ambiente Experimental}\label{sec:ambiente-experimental}

\lstdefinestyle{terminal-abnt}{
    language={},                        
    backgroundcolor=\color{listbggray}, 
    basicstyle=\ttfamily\small,         
    keywordstyle=,                      % Estilo de keyword (vazio)
    stringstyle=,                       % Estilo de string (vazio)
    numbers=none,                       % Números de linha à esquerda
    numberstyle=\tiny,                  % Estilo do número (sem cor)
    numbersep=5pt,                      % Distância dos números
    frame=none,                         % Sem bordas (como no seu main.pdf)
    captionpos=t,                       % Posição da legenda (t = top), como no seu main.pdf
    breaklines=true,                    % Quebra linhas longas
    breakatwhitespace=true,             % Quebra apenas em espaços
    showstringspaces=false,             % Não mostra símbolos para espaços
    tabsize=2                           % Tamanho do TAB
}
\newtcblisting{abntbox}{
    enhanced,
    listing only,
    colback=listbggray,     
    colframe=listbggray,    
    arc=3mm,                
    left=2mm,
    right=2mm,
    top=2mm,
    bottom=2mm,
    fonttitle=\bfseries,
    coltitle=black,         
    listing options={style=terminal-abnt}
}

Esta seção detalha os passos necessários para configurar o ambiente de desenvolvimento, isto envolve compilar o fork do Robcmp e replicar o sistema de build utilizado nos experimentos. 

No futuro, um novo desenvolvedor não precisará seguir estes passos pois o Robcmp já disponibilizará todas as possíveis bibliotecas pré-compiladas do \textit{wrapper} e do TFLM. Assim, o processo de desenvolvimento poderá ser feito através de forma automática e integrada ao PlatformIO, da mesma forma que já ocorre para programas embarcados convencionais. 

Porém, esta seção é ainda é valiosa por ensinar o processo de instalação dos pacotes necessários, clonagem correta do repositório e compilação do Robcmp, tarefas obrigatórias para um novo desenvolvedor. Além disso, esta seção é útil para trabalhos futuros que desejem modificar o compilador Robcmp ou entender melhor sua integração com o TFLM.

\subsection{Pré-requisitos de Software}
Antes de iniciar, é necessário garantir que os seguintes pacotes de software estejam instalados no sistema (preferencialmente um ambiente Linux, como o Ubuntu):

\begin{itemize}
    \item Git: Para controle de versão e download do repositório.
    \item CMake: Para configurar o sistema de build do Robcmp.
    \item Make: Para executar os scripts de compilação.
    \item LLVM (versão 20+): Conjunto de ferramentas e bibliotecas para compilação.
    \item Clang (versão 20+): Compilador C/C++ baseado no LLVM.
    \item Flex (versão 2.6.4 ou similar): Gerador de analisador léxico.
    \item Bison (versão 3.8.2 ou similar): Gerador de analisador sintático.
\end{itemize}

A maioria desses pacotes pode ser instalada em sistemas baseados em Debian/Ubuntu com os comandos:

% \begin{lstlisting}[style=terminal-abnt]
% sudo apt-get update
% sudo apt-get install git cmake make g++ flex bison
% \end{lstlisting}

\begin{abntbox}
sudo apt-get update
sudo apt-get install git cmake make clang llvm flex bison
\end{abntbox}

\subsection{Download do Repositório e Submódulos}
O fork do Robcmp utiliza submódulos do Git para gerenciar a dependência do TFLM. É necessário iniciar o submódulo existente após o clone com os seguintes comandos:

\begin{abntbox}
# Clona o repositorio fork do Robcmp
git clone https://github.com/LuizEduardoRezende/robcmp.git

# Navega ate o diretorio do repositorio
cd robcmp

# Seleciona a branch com o frontend do TFLM
git checkout tflm-front-end

# Inicializa e atualiza os submodulos
git submodule update --init --recursive
\end{abntbox}

Talvez seja necessário atualizar o seu explorador de arquivos para vizualiar os novos arquivos criados, após a inicialização e atualização dos submódulos.

\subsection{Processo de Compilação (\textit{Build})}
O processo de compilação é dividido em duas etapas, conforme a arquitetura descrita na \hyperref[sec:arq-camada-compatibilidade]{Seção~\ref{sec:arq-camada-compatibilidade}}. Primeiro, compila-se a biblioteca estática do TFLM; em seguida, compila-se o Robcmp (que inclui o \textit{wrapper} C).

Caso esteja buscando compilar para uma arquitetura especifica (ex: Cortex-M4), é possível verificar todas as arquiteturas suportadas pelo TFLM na documentação do \textit{GitHub} \footnote{\url{https://github.com/tensorflow/tflite-micro/blob/main/tensorflow/lite/micro/cortex_m_generic/README.md}}.

Com uma arquitetura específica em mente, navegue até o diretório do submódulo do TFLM, depois use o sistema de Makefile do TFLM para compilar a biblioteca para a arquitetura desejada:

\begin{abntbox}
# Navega ate o diretorio do TFLM
cd third_party/tflite-micro/

# Compila para uma plataforma especifica
make -f tensorflow/lite/micro/tools/make/Makefile TARGET=cortex_m_generic TARGET_ARCH=cortex-m4 microlite

# Compila para a plataforma de sua maquina de desenvolvimento(linux, windows, etc)
make -f tensorflow/lite/micro/tools/make/Makefile microlite
\end{abntbox}

Ao final deste processo, a biblioteca estática com nome padronizado deverá estar presente no diretório \texttt{third-party/\allowbreak tflite-micro/\allowbreak gen/\allowbreak (nome-da-arquitetura)/\allowbreak lib/\allowbreak libtensorflow-microlite.a}. Talvez seja necessário atualizar o seu explorador de arquivos para vizualiar os novos arquivos criados.

Atenção, caso tenha feito a compilação de mais de uma biblioteca estática do TFLM, escolha apenas uma arquitetura para trabalhar por vez. Para que o build e a configuração do ambiente de desenvolvimento ocorram corretamente é necessário que exista apenas uma arquitetura dentro de \texttt{third-party/tflite-micro/gen}. Caso deseje trabalhar com multiplas arquiteturas, é recomendado que se crie cópias do repositório do Robcmp para cada arquitetura desejada.

Com a biblioteca TFLM pronta, retorne ao diretório raiz do robcmp para compilar o projeto principal utilizando o CMake e o Make:

\begin{abntbox}
# Cria o diretorio de build e navega ate ele
mkdir build
cd build

# Configura o projeto
cmake ..

# Compila o projeto (use -j para compilacao paralela, ex: -j6)
make
\end{abntbox}

Se a compilação for bem-sucedida, o executável do compilador (arquivo \texttt{robcmp}) estará disponível dentro do diretório \texttt{build}, pronto para ser usado nos testes de validação.

\subsection{Compilação de arquivos .rob que usam o TFLM (\texttt{Makefile.tflm})}

Para conseguir compilar um programa \texttt{.rob} criado ou já existente basta utilizar o \texttt{Makefile.tflm} localizado na raíz do Robcmp. Este Makefile consegue compilar programas que estejam localizados dentro da pasta \texttt{test/tflm-tests}. A partir da raíz do projeto Robcmp, utilize os seguintes comandos:

\begin{abntbox}
# Mostra todos os comandos disponiveis 
make -f Makefile.tflm help

# Compila e gera o executavel na pasta build
make -f Makefile.tflm FILE=syntax-spam

# Remove os arquivos gerados desse programa
make -f Makefile.tflm FILE=syntax-spam clean

# Remove os arquivos objeto(.o) gerados por programas
make -f Makefile.tflm clean-all

# Executa o programa que foi gerado na pasta build
./build/syntax-spam
\end{abntbox}

\section{Resultados da Validação Funcional}
Como o objetivo da validação funcional era garantir que as novas funcionalidades implementadas estivessem operando conforme o esperado, foram criados diversos arquivos \texttt{.rob} que utilizavam diferentes modelos de ML, abrangendo uma variedade de cenários e casos de uso.

A maioria dos testes unitários do Robcmp estão presentes na pasta \texttt{test/general} e são executados automaticamente através do Makefile e de um script \texttt{run-tests.sh}. No caso dos testes unitários relacionados ao TFLM, estes estão localizados na pasta \texttt{test/tflm-tests}. Essa separação foi necessária pois o processo de linkedição dos programas que utilizam o TFLM é diferente dos programas convencionais, exigindo a inclusão da biblioteca estática do TFLM e do \textit{wrapper} C.

Dentro do diretório de testes desejado, basta executar o comando \texttt{make} para que todos os testes sejam compilados e executados automaticamente. Foram desenvolvidos testes unitários tanto utilizando a nova sintaxe da RL quanto utilizando a biblioteca padrão do Robcmp para TinyML. Como convenção, os arquivos de teste que utilizam a nova sintaxe possuem o prefixo \texttt{syntax-}, enquanto os que utilizam a biblioteca padrão possuem o prefixo \texttt{lib-}.

Dois arquivos de exemplo foram colocados nos apêndices no final deste trabalho que realizam os testes unitários com o modelo de classificação de spam. O \hyperref[]{Código~\ref*{}} apresenta o arquivo \texttt{syntax-spam.rob}, que utiliza a nova sintaxe da RL, enquanto o \hyperref[lst:teste-bib-rob]{Código~\ref*{lst:teste-bib-rob}} apresenta o arquivo \texttt{lib-spam.rob}, que utiliza a biblioteca padrão.

Esses testes verificam se o modelo de ML é carregado corretamente, se as entradas são processadas adequadamente e se as saídas correspondem ao esperado, entre outros aspectos. Ambos os arquivos foram compilados e executados com sucesso, confirmando o correto funcionamento das novas funcionalidades.

\section{Resultados da Validação da Eficiência e Análise Comparativa}


\begin{table}[h]
\centering 
\caption{Tamanho do código objeto em \textit{bytes}}
\label{comparativo}
    \begin{tabular}{cccc}
    \toprule
    \textbf{Modelo TinyML} & \textbf{Sintaxe} & \textbf{Biblioteca padrão} & \textbf{TFLM(C++)}  \\ \midrule
    Classificador de Spam   \\ 
    Preditor de Seno quantizado \\
    Preditor de Seno  \\ 
    \bottomrule
    
    \end{tabular}
\end{table}

\begin{table}[h]
\centering 
\caption{Tempo de execução, em milissegundos}
\label{comparativo}
    \begin{tabular}{cccc}
    \toprule
    \textbf{Modelo TinyML} & \textbf{Sintaxe} & \textbf{Biblioteca padrão} & \textbf{TFLM(C++)}  \\ \midrule
    Classificador de Spam   \\ 
    Preditor de Seno quantizado \\
    Preditor de Seno  \\ 
    \bottomrule
    
    \end{tabular}
\end{table}

\section{Considerações Finais}




\chapter{Conclusões e Trabalhos Futuros}\label{cap:conclusao}

\section{Conclusões}
Este trabalho realizou a extensão da RL através da integração da biblioteca TFLM. Essa extensão ocorreu de duas formas distintas: uma concluída e a outra parcialmente implementada. A primeira foi a criação da biblioteca padrão \texttt{ai.tflm} para o Robcmp; já a segunda foi a proposta e o planejamento da adaptação do \textit{frontend} do compilador para suportar novas palavras-chave e tipos de dados, permitindo a escrita de programas em RL que utilizem o TFLM de forma nativa.

O desenvolvimento da biblioteca \texttt{ai.tflm} foi bem-sucedido. Ela utiliza a camada de interoperabilidade explicada na \hyperref[sec:arq-camada-compatibilidade]{Seção~\ref*{sec:arq-camada-compatibilidade}} e o \textit{wrapper} desenvolvido na \hyperref[sec:wrapper]{Seção~\ref*{sec:wrapper}} para expor as funcionalidades do TFLM ao Robcmp. A validação ocorreu na máquina de desenvolvimento por meio de testes unitários, garantindo funcionalidade e confiabilidade para diferentes modelos de TinyML. Esses modelos processam diversos tipos de entrada, como números simples, texto tokenizado e a representação em arrays de bytes de imagens e espectrogramas de áudio. 

A eficiência da solução também foi avaliada em termos de tamanho do binário e tempo de execução na \hyperref[sec:resultados-analise-comparativa]{Seção~\ref*{sec:resultados-analise-comparativa}}. Apesar do aumento no tamanho do binário e no tempo de execução, os impactos foram considerados aceitáveis, especialmente quando comparados aos benefícios em termos de facilidade de desenvolvimento e manutenibilidade do código.

A adaptação do \textit{frontend} do Robcmp para suportar o TFLM nativamente não foi concluída. No entanto, grande parte já foi implementada e detalhada durante o texto. A implementação dessa funcionalidade permitiria aos desenvolvedores escrever programas em RL que utilizem o TFLM de forma mais intuitiva, sem a necessidade de recorrer à biblioteca \texttt{ai.tflm}. Essa abordagem nativa facilitaria a integração de modelos TinyML em aplicações RL, promovendo uma experiência de desenvolvimento mais fluida e eficiente.

Pode-se concluir que o desenvolvimento do \textit{wrapper} e da biblioteca \texttt{ai.tflm} representa uma base sólida para o desenvolvimento de aplicações TinyML em RL. Isso permite que aplicações de ML sejam desenvolvidas para dispositivos embarcados com recursos limitados, aproveitando as capacidades do TFLM e também da RL que inclui desenvolvimento de \textit{firmware} com baixo acoplamento, alta coesão, elevada manutenibilidade e abstração de \textit{hardware}.

O trabalho tem como objetivo inicial o suporte à arquitetura ARM Cortex-M, utilizada por diversas famílias de microcontroladores (como a STM32). No entanto, o desenvolvedor pode escolher a arquitetura desejada para gerar o \textit{firmware}, desde que o Robcmp e o TFLM ofereçam suporte à plataforma escolhida. Para isso, é necessário compilar tanto o wrapper quanto a biblioteca do TFLM para a arquitetura específica, garantindo a portabilidade do sistema.

\section{Limitações}
Apesar do sucesso na implementação, algumas partes apresentam certas limitações que, devido ao tempo disponível e ao grande escopo do projeto, não puderam ser totalmente aprimoradas:

\begin{itemize}
    \item \textbf{Ausência de Testes em Ambiente Embarcado:} O plano inicial incluía a realização de testes em uma placa embarcada (STM32F407VET6). Contudo, devido a limitações de tempo, os testes em ambiente real não foram concluídos. Recomenda-se que futuros trabalhos explorem essa etapa.
    \item \textbf{Perda de Precisão na Quantização:} Uma das possíveis abordagens para a entrada e saída de dados exige que o wrapper realize a conversão e quantização de variáveis ou vetores de ponto flutuante. Embora simplifique o uso, esse processo pode introduzir perda de precisão ao utilizar métodos de arredondamento, o que é um fator de atenção em aplicações sensíveis a erros.
    \item \textbf{Restrição da Sintaxe de Acesso a Tensores:} Os tokens que tratam o acesso aos tensores de entrada e saída exigem uma sintaxe específica, impedindo que campos com os nomes \texttt{input} e \texttt{output} sejam declarados em tipos definidos pelo usuário. O ideal seria tratar o acesso aos tensores como membros normais de um tipo.
    \item \textbf{Baixa Modularidade do \texttt{ModelNode}:} A classe \texttt{ModelNode} concentra uma grande quantidade de responsabilidades e utiliza um esquema de roteamento interno para decidir qual operação realizar, resultando em um código menos modular e mais acoplado.
    \item \textbf{Implementação Não Concluída do Frontend Nativo:} A adaptação do frontend do Robcmp para suportar o TFLM nativamente não foi concluída, o que exigiu o uso da biblioteca padrão. Sua finalização é essencial para uma experiência de desenvolvimento mais fluida e intuitiva.
    
\end{itemize}

\section{Trabalhos Futuros}
Com base nas contribuições e nas limitações deste trabalho, as seguintes melhorias e caminhos de pesquisa são sugeridos para o compilador Robcmp e sua linguagem RL, tanto no contexto de TinyML quanto em outras áreas:

\begin{itemize}
    \item Finalização da implementação do \textit{frontend} do Robcmp para suportar o TFLM nativamente, conforme planejado.
    \item Validação da biblioteca \texttt{ai.tflm} e do \textit{frontend} adaptado em dispositivos embarcados reais, avaliando seu desempenho e eficiência em ambientes com recursos limitados.
    \item Exploração de outras bibliotecas de ML para integração com o Robcmp, ampliando as opções disponíveis para desenvolvedores.
    \item Criação de novas estratégias no \textit{wrapper} desenvolvido com o intuito de: modernizar a otimização de tamanho de binário e contornar a perda de precisão durante a quantização e conversão de tipos.
\end{itemize}

Com base nas contribuições deste trabalho, espera-se que o Robcmp e a linguagem RL possam continuar a evoluir, incorporando novas funcionalidades e melhorias que atendam às necessidades dos desenvolvedores de sistemas embarcados e aplicações de TinyML. A integração do TFLM é um passo importante nessa direção, e futuras pesquisas podem explorar ainda mais as possibilidades em áreas como TinyML, \textit{Edge AI} e sistemas IoT inteligentes.

% \section{Conclusões}
% As conclusões devem estabelecer uma descrição sucinta e sintética daquilo que o autor concluiu ao desenvolver sua pesquisa. Deve haver um cuidado para que a mesma não seja óbvia e também que não seja impossível de identificar no texto. Conclusões sobre um ou outro tipo de tecnologia usada, conclusões sobre caminhos que foram tomados na condução da pesquisa são importantes. E, cabe ressaltar que este texto é do autor, portanto não cabe nesta seção a inserção de referências bibliográficas.

% \subsection{Quanto à área aplicada}
% \subsection{Quanto à área específica}

% \section{Trabalhos futuros}
% Em todo trabalho científico, vários caminhos podem ser estabelecidos. Porém cabe geralmente ao autor definir um único para viabilizar a produção e divulgação da sua pesquisa. Estes outros caminhos podem ser apresentados nesta seção, detalhando claramente os motivos da não escolha pelos mesmos. 
% Também muito importante nesta seção, é vislumbrar o que ainda pode ser realizado na sequência do próprio trabalho. Toda pesquisa é provavelmente infinita, o que a classifica como concluída, é apenas um ponto de parada para sua divulgação. Portanto, outras contribuições são e serão sempre passíveis. É exatamente isso que cria a evolução, o desenvolvimento e gera inovação. Na área da Ciências Exatas, o termo “Estado da Arte” dá a exatidão desta sequencia. 






% ----------------------------------------------------------
% ELEMENTOS PÓS-TEXTUAIS
% ----------------------------------------------------------
\postextual
% ----------------------------------------------------------

% ----------------------------------------------------------
% Referências bibliográficas
% ----------------------------------------------------------
\bibliography{bib}

% ----------------------------------------------------------
% Apêndices
% ----------------------------------------------------------


% ----------------------------------------------------------
% Anexos
% ----------------------------------------------------------

% ---
% Inicia os anexos
% ---
% ---
% Inicia os anexos
% ---
\begin{anexosenv}

% Imprime uma página indicando o início dos anexos
\partanexos

\chapter{Exemplo de Formulário}\label{anexo:formAluno}

Este formulário tem como objetivo ...:

Pergunta 1:

Pergunta 2:

Pergunta 3:


\end{anexosenv}

%---------------------------------------------------------------------
% INDICE REMISSIVO
%---------------------------------------------------------------------
%\phantompart
\printindex
%---------------------------------------------------------------------

\end{document}
