% ---
% RESUMOS
% ---

% resumo em português
\setlength{\absparsep}{18pt} % ajusta o espaçamento dos parágrafos do resumo
\begin{resumo}

Resumo na língua vernácula, elaborado em folha separada, deve indicar, concisamente, os pontos relevantes do trabalho: objeto de estudo, problema, tema objetivos, justificativas,  metodologia, resultados esperados ou obtidos, o valor científico do trabalho e sua originalidade, contendo, no máximo 400 palavras e no mínimo 200. Deve ser composto de uma sequência de frases concisas e não deve ser elaborado na forma de tópicos. Deve ser seguido das palavras-chave ou descritores, precedido de dois espaços simples, tendo no mínimo de 3 e no máximo de 5 palavras, isto é, inserir palavras que mais representam o conteúdo do trabalho, ressaltando que as mesmas devem vir separadas por ponto. O resumo deve ser em texto corrido e sem parágrafo, digitado em espaço simples. 

 \textbf{Palavras-chaves}: \textit{PalavraChave1;Palavra-Chave2; Palavra-Chave3.} 


\end{resumo}

% resumo em inglês
\begin{resumo}[Abstract]
 \begin{otherlanguage*}{english}
 Abstract (Resumo na língua estrangeira),contém as orientações do resumo em língua estrangeira, digitado em folha separada. Por ser elaborado em idioma de comunicação internacional (inglês – Abstract). Deve, também, ser seguido das palavras-chave (inglês - Keywords). Os critérios de formatação do Abstract seguem os mesmos do Resumo na língua vernácula.  
   
   \textbf{Key-words}: 
   \textit{Keyword1; Keyword2; Keyword3}.
 \end{otherlanguage*}
\end{resumo}

