\chapter{Avaliação e Testes}\label{cap:analise}

\section{Introdução}
Neste capítulo é apresentada uma análise da aplicação ou uso do que foi construído e demonstrado no capítulo anterior. Esta análise pode ser realizada usando normas (ISO/EIC) já estabelecidas. Pode ser por meio de testes de laboratório, pode ser simulada, pode ser aplicada em um contexto (universo/população/amostragem) real. E, também pode ser por meio da criação de uma metodologia própria para a análise.

\section{Ambiente Experimental}

\lstdefinestyle{terminal-abnt}{
    language={},                        
    backgroundcolor=\color{listbggray}, 
    basicstyle=\ttfamily\small,         
    keywordstyle=,                      % Estilo de keyword (vazio)
    stringstyle=,                       % Estilo de string (vazio)
    numbers=none,                       % Números de linha à esquerda
    numberstyle=\tiny,                  % Estilo do número (sem cor)
    numbersep=5pt,                      % Distância dos números
    frame=none,                         % Sem bordas (como no seu main.pdf)
    captionpos=t,                       % Posição da legenda (t = top), como no seu main.pdf
    breaklines=true,                    % Quebra linhas longas
    breakatwhitespace=true,             % Quebra apenas em espaços
    showstringspaces=false,             % Não mostra símbolos para espaços
    tabsize=2                           % Tamanho do TAB
}
\newtcblisting{abntbox}{
    enhanced,
    listing only,
    colback=listbggray,     
    colframe=listbggray,    
    arc=3mm,                
    left=2mm,
    right=2mm,
    top=2mm,
    bottom=2mm,
    fonttitle=\bfseries,
    coltitle=black,         
    listing options={style=terminal-abnt}
}

Esta seção detalha os passos necessários para configurar o ambiente de desenvolvimento, compilar o fork do Robcmp e replicar o sistema de build utilizado nos experimentos.

\subsection{Pré-requisitos de Software}
Antes de iniciar, é necessário garantir que os seguintes pacotes de software estejam instalados no sistema (preferencialmente um ambiente Linux, como o Ubuntu):

\begin{itemize}
    \item Git: Para controle de versão e download do repositório.
    \item Cmake: Para configurar o sistema de build do Robcmp.
    \item Make: Para executar os scripts de compilação.
    \item Clang: Um compilador C++ moderno.
    \item Flex: (versão 2.6.4 ou similar): Gerador de analisador léxico.
    \item Bison: (versão 3.8.2 ou similar): Gerador de analisador sintático.
\end{itemize}

A maioria desses pacotes pode ser instalada em sistemas baseados em Debian/Ubuntu com os comandos:

\begin{lstlisting}[style=terminal-abnt]
sudo apt-get update
sudo apt-get install git cmake make g++ flex bison
\end{lstlisting}

\begin{abntbox}
sudo apt-get update
sudo apt-get install git cmake make g++ flex bison
\end{abntbox}

\subsection{Download do Repositório e Submódulos}
O fork do Robcmp utiliza submódulos do Git para gerenciar a dependência do TFLM. É necessário iniciar o submódulo existente após o clone com os seguintes comandos:

\begin{abntbox}
# Clona o repositorio fork do Robcmp
git clone https://github.com/LuizEduardoRezende/robcmp.git

# Navega ate o diretorio do repositorio
cd robcmp

# Seleciona a branch com o frontend do TFLM
git checkout tflm-front-end

# Inicializa e atualiza os submodulos
git submodule update --init --recursive
\end{abntbox}

Talvez seja necessário atualizar o seu explorador de arquivos para vizualiar os novos arquivos criados, após a inicialização e atualização dos submódulos.

\subsection{Processo de Compilação (\textit{Build})}
O processo de compilação é dividido em duas etapas, conforme a arquitetura descrita na \hyperref[sec:arq-camada-compatibilidade]{Seção~\ref{sec:arq-camada-compatibilidade}}. Primeiro, compila-se a biblioteca estática do TFLM; em seguida, compila-se o Robcmp (que inclui o \textit{wrapper} C).

Caso esteja buscando compilar para uma arquitetura especifica (ex: Cortex-M4), é possível verificar todas as arquiteturas suportadas pelo TFLM na documentação do \textit{GitHub} \footnote{\url{https://github.com/tensorflow/tflite-micro/blob/main/tensorflow/lite/micro/cortex_m_generic/README.md}}.

Com uma arquitetura específica em mente, navegue até o diretório do submódulo do TFLM, depois use o sistema de Makefile do TFLM para compilar a biblioteca para a arquitetura desejada:

\begin{abntbox}
# Navega ate o diretorio do TFLM
cd third_party/tflite-micro/

# Compila para uma plataforma especifica
make -f tensorflow/lite/micro/tools/make/Makefile TARGET=cortex_m_generic TARGET_ARCH=cortex-m4 microlite

# Compila para a plataforma de sua maquina de desenvolvimento(linux, windows, etc)
make -f tensorflow/lite/micro/tools/make/Makefile microlite
\end{abntbox}

Ao final deste processo, a biblioteca estática com nome padronizado deverá estar presente no diretório \texttt{third-party/\allowbreak tflite-micro/\allowbreak gen/\allowbreak (nome-da-arquitetura)/\allowbreak lib/\allowbreak libtensorflow-microlite.a}. Talvez seja necessário atualizar o seu explorador de arquivos para vizualiar os novos arquivos criados.

Atenção, caso tenha feito a compilação de mais de uma biblioteca estática do TFLM, escolha apenas uma arquitetura para trabalhar por vez. Para que o build e a configuração do ambiente de desenvolvimento ocorram corretamente é necessário que exista apenas uma arquitetura dentro de \texttt{third-party/tflite-micro/gen}. Caso deseje trabalhar com multiplas arquiteturas, é recomendado que se crie cópias do repositório do Robcmp para cada arquitetura desejada.

Com a biblioteca TFLM pronta, retorne ao diretório raiz do robcmp para compilar o projeto principal utilizando o Cmake e o Make:

\begin{abntbox}
# Cria o diretorio de build e navega ate ele
mkdir build
cd build

# Configura o projeto
cmake ..

# Compila o projeto (use -j para compilacao paralela, ex: -j6)
make
\end{abntbox}

Se a compilação for bem-sucedida, o executável do compilador (robcmp) estará disponível dentro do diretório \texttt{build}, pronto para ser usado nos testes de validação.

\subsection{Compilação de arquivos .rob que usam o TFLM (\texttt{Makefile.tflm})}

Para conseguir compilar um programa \texttt{.rob} criado ou já existente basta utilizar o \texttt{Makefile.tflm} localizado na raíz do Robcmp. Este Makefile consegue compilar programas que estejam localizados dentro da pasta \texttt{test/tflm-tests}. A partir da raíz do projeto Robcmp, utilize os seguintes comandos:

\begin{abntbox}
# Mostra todos os comandos disponiveis 
make -f Makefile.tflm help

# Compila e gera o executavel na pasta build
make -f Makefile.tflm FILE=syntax-spam

# Remove os arquivos gerados desse programa
make -f Makefile.tflm FILE=syntax-spam clean

# Remove os arquivos objeto(.o) gerados por programas
make -f Makefile.tflm clean-all

# Executa o programa que foi gerado na pasta build
./build/syntax-spam
\end{abntbox}


\section{Validação Funcional}

\section{Validação da eficiência e Análise Comparativa}

\section{Considerações Finais}



